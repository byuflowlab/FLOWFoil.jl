\section{Singularity Distributions}
\label{sec:singularitydistributions}

%TODO: explain what a line distribution is first as an intro before getting into panel specific stuff.

As will be discussed in greater detail in \cref{chp:bem}, the applications of potential flow theory presented in this dissertation are based on finding flow disturbances due to solid bodies.  
We approximate (or ideally represent exactly) arbitrary body geometries with panels.  
Here we will limit our discussion to a single panel from which we can glean quite a bit of understanding that will be applied in subsequent chapters.

\begin{marginfigure}
	\centering
	\begin{tikzpicture}[scale=1.0]
		\draw[ultra thick, color=white] (0,-0.5) -- (0,2);
		\draw[ultra thick,color=navy,mark=square*] plot coordinates {(-1.5,-0.10)} -- plot[mark=square*] coordinates {(1.5,.25)};
		\draw[color=plotsred,fill=plotsred] (0.2,1.25) circle (.5ex);
		\draw[color=plotsred] (0.5,1.5) node {$\mathbf{P}$};
		\draw[thick, color=plotsgray] (-0.9,-0.15) -- (-0.9,0.1);
		\draw[color=plotsgray] (-0.65,-0.3) node {${ds}$};
		\draw[decorate,decoration={brace,amplitude=10pt},yshift=2pt,xshift=-3pt] (-0.85,-0.05) -- (0.2,1.25) node [black,midway,xshift=-11pt,yshift=8pt] {$r$};
		%		\draw[ultra thick, color=black] (-0.9,0) -- (0.2,1.25);
	\end{tikzpicture}
\end{marginfigure}



Let us begin with a single, flat, two-dimensional ``panel'' of finite length and with arbitrary orientation, and some arbitrary point in space \(\mathbf{P}\). 
In general terms, the potential at some point of interest due to an arbitrary distribution of singularities along the panel is

\[\phi(\mathbf{P}) = \int_s \phi_s(r) \d s\]

\noindent where \(s\) is the curvilinear length along the panel, \(\phi_s(r)\) is the formula for the given singularity distributed along the panel, and \(r\) is the distance from the point, \(s\), along the panel to the point of interest.





\subsection{Source Distributions}

For potential sources, which are defined as \(\phi_s(r) = \frac{q(s)}{2\pi}\ln(r)\) this yields the potential due to a source distribution of

\[\phi_{source}(\mathbf{P}) =  \int_s \frac{q(s)}{2\pi} \ln(r) \d s\]

\noindent where \(q(s)\) is the source strength distribution along the panel as a function of \(s\).



\subsubsection{Constant Strength Distribution}

\begin{marginfigure}
	\begin{tikzpicture}[/tikz/background rectangle/.style={fill={rgb,1:red,1.0;green,1.0;blue,1.0}, draw opacity={1.0}}, show background rectangle]
\begin{axis}[point meta max={nan}, point meta min={nan}, title={}, title style={at={{(0.5,1)}}, anchor={south}, font={{\fontsize{14 pt}{18.2 pt}\selectfont}}, color={rgb,1:red,0.0;green,0.0;blue,0.0}, draw opacity={1.0}, rotate={0.0}}, legend style={color={rgb,1:red,0.0;green,0.0;blue,0.0}, draw opacity={0.0}, line width={1}, solid, fill={rgb,1:red,0.0;green,0.0;blue,0.0}, fill opacity={0.0}, text opacity={1.0}, font={{\fontsize{8 pt}{10.4 pt}\selectfont}}, text={rgb,1:red,0.0;green,0.0;blue,0.0}, cells={anchor={center}}, at={(1.02, 1)}, anchor={north west}}, axis background/.style={fill={rgb,1:red,0.0;green,0.0;blue,0.0}, opacity={0.0}}, anchor={north west}, xshift={1.0mm}, yshift={-1.0mm}, width={48.8mm}, height={48.8mm}, scaled x ticks={false}, xlabel={s}, x tick style={color={rgb,1:red,0.0;green,0.0;blue,0.0}, opacity={1.0}}, x tick label style={color={rgb,1:red,0.0;green,0.0;blue,0.0}, opacity={1.0}, rotate={0}}, xlabel style={at={(ticklabel cs:0.5)}, anchor=near ticklabel, font={{\fontsize{11 pt}{14.3 pt}\selectfont}}, color={rgb,1:red,0.0;green,0.0;blue,0.0}, draw opacity={1.0}, rotate={0.0}}, xmajorticks={false}, xmajorgrids={false}, xmin={0}, xmax={2}, axis x line*={left}, x axis line style={color={rgb,1:red,0.0;green,0.0;blue,0.0}, draw opacity={1.0}, line width={1}, solid}, scaled y ticks={false}, ylabel={$q(s)=$ const}, y tick style={color={rgb,1:red,0.0;green,0.0;blue,0.0}, opacity={1.0}}, y tick label style={color={rgb,1:red,0.0;green,0.0;blue,0.0}, opacity={1.0}, rotate={0}}, ylabel style={at={(ticklabel cs:0.5)}, anchor=near ticklabel, font={{\fontsize{11 pt}{14.3 pt}\selectfont}}, color={rgb,1:red,0.0;green,0.0;blue,0.0}, draw opacity={1.0}, rotate={0.0}}, ymajorticks={false}, ymajorgrids={false}, ymin={0}, ymax={3}, axis y line*={left}, y axis line style={color={rgb,1:red,0.0;green,0.0;blue,0.0}, draw opacity={1.0}, line width={1}, solid}, colorbar={false}]
    \addplot+[line width={0}, draw opacity={0}, fill={rgb,1:red,0.0;green,0.1804;blue,0.3647}, fill opacity={0.125}, mark={none}, forget plot]
        coordinates {
            (0.0,1.25)
            (1.75,1.25)
            (1.75,0.0)
            (0.0,0.0)
            (0.0,1.25)
        }
        ;
    \addplot[color={rgb,1:red,0.0;green,0.1804;blue,0.3647}, name path={274b047d-91ec-4240-8a0f-dfc54936dafa}, area legend, draw opacity={1.0}, line width={2.0}, solid]
        table[row sep={\\}]
        {
            \\
            0.0  1.25  \\
            1.75  1.25  \\
        }
        ;
\end{axis}
\end{tikzpicture}

\end{marginfigure}


If \(q(s) = {\color{navy}q_0} = \mathrm{const}\) is constant, we can pull the source strength out of the integral, simplifying the calculation:

\begin{equation}
\label{eqn:constantsourcedist}
	\phi_{source}(\mathbf{P}) = \frac{\color{navy}q_0}{2\pi} \int_s \ln(r) \d s
\end{equation}

\noindent For distributions of higher order than constant, we must leave the source strength inside the integral, which can require extra considerations when calculating [probably reference numeric integration stuff later].


\subsubsection{Linear Strength Distribution}

For example, when \(q(s)\) is distributed linearly we end up with

\[\phi_{source}(\mathbf{P}) = \frac{1}{2\pi} \int_s ({\color{plotsred}q_1}s + {\color{navy}q_0}) \ln(r) \d s.\]

\begin{marginfigure}
	\begin{tikzpicture}[/tikz/background rectangle/.style={fill={rgb,1:red,1.0;green,1.0;blue,1.0}, draw opacity={1.0}}, show background rectangle]
\begin{axis}[point meta max={nan}, point meta min={nan}, title={}, title style={at={{(0.5,1)}}, anchor={south}, font={{\fontsize{14 pt}{18.2 pt}\selectfont}}, color={rgb,1:red,0.0;green,0.0;blue,0.0}, draw opacity={1.0}, rotate={0.0}}, legend style={color={rgb,1:red,0.0;green,0.0;blue,0.0}, draw opacity={0.0}, line width={1}, solid, fill={rgb,1:red,0.0;green,0.0;blue,0.0}, fill opacity={0.0}, text opacity={1.0}, font={{\fontsize{8 pt}{10.4 pt}\selectfont}}, text={rgb,1:red,0.0;green,0.0;blue,0.0}, cells={anchor={center}}, at={(1.02, 1)}, anchor={north west}}, axis background/.style={fill={rgb,1:red,0.0;green,0.0;blue,0.0}, opacity={0.0}}, anchor={north west}, xshift={1.0mm}, yshift={-1.0mm}, width={48.8mm}, height={48.8mm}, scaled x ticks={false}, xlabel={s}, x tick style={color={rgb,1:red,0.0;green,0.0;blue,0.0}, opacity={1.0}}, x tick label style={color={rgb,1:red,0.0;green,0.0;blue,0.0}, opacity={1.0}, rotate={0}}, xlabel style={at={(ticklabel cs:0.5)}, anchor=near ticklabel, font={{\fontsize{11 pt}{14.3 pt}\selectfont}}, color={rgb,1:red,0.0;green,0.0;blue,0.0}, draw opacity={1.0}, rotate={0.0}}, xmajorticks={false}, xmajorgrids={false}, xmin={0}, xmax={2}, axis x line*={left}, x axis line style={color={rgb,1:red,0.0;green,0.0;blue,0.0}, draw opacity={1.0}, line width={1}, solid}, scaled y ticks={false}, ylabel={$q(s)=$ linear}, y tick style={color={rgb,1:red,0.0;green,0.0;blue,0.0}, opacity={1.0}}, y tick label style={color={rgb,1:red,0.0;green,0.0;blue,0.0}, opacity={1.0}, rotate={0}}, ylabel style={at={(ticklabel cs:0.5)}, anchor=near ticklabel, font={{\fontsize{11 pt}{14.3 pt}\selectfont}}, color={rgb,1:red,0.0;green,0.0;blue,0.0}, draw opacity={1.0}, rotate={0.0}}, ymajorticks={false}, ymajorgrids={false}, ymin={0}, ymax={3}, axis y line*={left}, y axis line style={color={rgb,1:red,0.0;green,0.0;blue,0.0}, draw opacity={1.0}, line width={1}, solid}, colorbar={false}]
    \addplot+[line width={0}, draw opacity={0}, fill={rgb,1:red,0.0;green,0.1804;blue,0.3647}, fill opacity={0.125}, mark={none}, forget plot]
        coordinates {
            (0.0,1.25)
            (1.75,1.25)
            (1.75,0.0)
            (0.0,0.0)
            (0.0,1.25)
        }
        ;
    \addplot[color={rgb,1:red,0.0;green,0.1804;blue,0.3647}, name path={fc7d24c6-17ae-46ac-8eaa-9c4ad9245a2c}, area legend, draw opacity={1.0}, line width={2.0}, solid]
        table[row sep={\\}]
        {
            \\
            0.0  1.25  \\
            1.75  1.25  \\
        }
        ;
    \addplot+[line width={0}, draw opacity={0}, fill={rgb,1:red,0.6078;green,0.0;blue,0.0}, fill opacity={0.125}, mark={none}, forget plot]
        coordinates {
            (0.0,1.25)
            (1.75,2.25)
            (1.75,1.25)
            (0.0,1.25)
            (0.0,1.25)
        }
        ;
    \addplot[color={rgb,1:red,0.6078;green,0.0;blue,0.0}, name path={c2a2d390-e979-4ed0-8601-00892f3dbde4}, area legend, draw opacity={1.0}, line width={2.0}, solid]
        table[row sep={\\}]
        {
            \\
            0.0  1.25  \\
            1.75  2.25  \\
        }
        ;
\end{axis}
\end{tikzpicture}

\end{marginfigure}

\noindent In this case, and subsequent higher orders, we can split out the integral, taking advantage of the simpler calculation for the constant portion of the integrand

\begin{equation}
\label{eqn:linearsourcedist}
	\phi_{source}(\mathbf{P}) = \frac{1}{2\pi} \left[ \int_s {\color{plotsred}q_1}s \ln(r) \d s + {\color{navy}q_0} \int_s \ln(r) \d s \right].
\end{equation}


\subsubsection{NURBS Strength Distribution}
\toadd{need a margin plot here}


\begin{equation}
\label{eqn:nurbssourcedist}
	\phi_{source}(\mathbf{P}) = \frac{1}{2\pi} \int_s \sum_{i=1}^n  \left[ R_{i,p}(s) q_i \right] \ln(r) \d s.
\end{equation}

\noindent where \(R_{i,p}\) is the \(i\)th NURBS basis function for the \(p\)th order NURBS curve\footnote{We'll typically be using p=3, or cubic splines in this dissertation.} associated with the analysis geometry, and \(q_i\) are the coefficients for the NURBS definition.
Due to the nature of the NURBS basis functions, the integral can be split, but not quite as cleanly as can be done for standard polynomials as mentioned above.
This will be explained more fully as we look at some examples.

\subsection{Examples: Single Panels}


\subsubsection{Linear Source Distribution}

\toremove{all the parentheses here are a mess.  figure out a notation that cleans them up.}

In the linear source distribution case, we will use \cref{eqn:linearsourcedist} to calculate the potential at point \(\mathbf{P}\).
As can be seen, we can split our integral into the constant and linear portions.
We'll start with the simpler, constant portion of the integral.

First, let us define \(\xi\) on the range \(0 \leq \xi \leq 1\), where \(\xi\) is a parametric value input into the definition of our panel.
That is to say, let us define our integral as

\[ \frac{q_0}{2\pi} \int_0^1 \ln(r(s(\xi))) J(s(\xi)) \d\xi \]

\noindent where

\[ r(s(\xi)) = || \mathbf{P} - s(\xi) ||, \]

\noindent \(s(\xi)\) is the position along our panel at the parametric point, \(\xi\), and 

\[J(s(\xi)) = \norm{\frac{\d s}{\d\xi}}\] 

\noindent is required for the transformation from \(\xi\) to \(s\) in the integration bounds.

Remember that our numerical integration methods require bounds (-1,1), so we will have to perform another transformation using \cref{eqn:linearboundstransform}.
Doing so gives

\[ \frac{q_0}{2\pi} \int_{-1}^1 \ln(r(s(\xi(t)))) J(s(\xi(t))) \d t, \]

\noindent where

\[J(s(\xi(t))) = \norm{\pd{ s}{ \xi}} \pd{ \xi}{ t} .\]

\noindent If the point of interest, \(\mathbf{P}\), is on the panel itself, we need to also apply the Monegato-Sloan transformation of \cref{eqn:mstransformint}:

\[ \frac{q_0}{2\pi} \int_{-1}^1 \ln(r(s(\xi(\gamma(t))))) ~J(s(\xi(\gamma(t)))) ~\d t, \]
%Now we'll move on to the linear portion of the integral.

\noindent where

\[ r(s(\xi(\gamma(t)))) = || s(\xi(\gamma(t_P))) - s(\xi(\gamma(t))) ||, \]

\noindent where \(t_P\) is the parametric value associated with \(\mathbf{P}\).

\noindent and

\[J(s(\xi(\gamma(t)))) = \norm{\pd{ s}{ \xi}} \pd{ \xi}{ \gamma} \pd{ \gamma}{ t}  .\]

We can now look at the linear portion of the integral.  It is nearly identical, with the exception that the linear strength coefficient, \(q_1\), is multiplied by the parametric distance along the curve:

\[ \frac{1}{2\pi} \int_{-1}^1 (q_1\xi(\gamma(t))) ~\ln(r(s(\xi(\gamma(t))))) ~J(s(\xi(\gamma(t)))) ~\d t, \]

\noindent All together, then we have the full integral of a linear distribution along an arbitrary panel:

\[
\begin{split}
	\phi_{source}(\mathbf{P}) =& \frac{1}{2\pi} \biggr[
	q_0 \int_{-1}^1 \ln(r(s(\xi(\gamma(t))))) ~J(s(\xi(\gamma(t)))) ~\d t \\
	&+ \int_{-1}^1 (q_1\xi(\gamma(t))) ~\ln(r(s(\xi(\gamma(t))))) ~J(s(\xi(\gamma(t)))) ~\d t \biggr].
\end{split}
\]

\noindent If there are no weak singularities to deal with, then we simply remove the Monegato-Sloan transformation terms.

\[
\begin{split}
	\phi_{source}(\mathbf{P}) =& \frac{1}{2\pi} \biggr[
	q_0 \int_{-1}^1 \ln(r(s(\xi(t)))) ~J(s(\xi(t))) ~\d t \\
	&+ \int_{-1}^1 (q_1\xi(t)) ~\ln(r(s(\xi(t)))) ~J(s(\xi(t))) ~\d t \biggr]
\end{split}
\]

\noindent where

\[ r(s(\xi(t))) = || s(\xi(t_P)) - s(\xi(t)) ||, \]

\noindent and

\[J(s(\xi(\gamma(t)))) = \norm{\pd{ s}{ \xi}} \pd{ \xi}{ t} .\]



\subsubsection{Curved Panel, NURBS Source Distribution}
\toadd{write this section}

