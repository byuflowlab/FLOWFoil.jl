\section{Preliminaries}
\label{sec:potentialflowpreliminaries}

\subsection{Elementary Flows}
\label{ssec:elemflows}

As we begin this section, it is important to note that the Laplace equation is a \textit{linear} equation.
We will be taking advantage of that fact moving forward, as it allows us to describe complex flow fields through the superposition of many, much simpler flows.
The following are a handful of useful elementary flow solutions.

\subsubsection{Uniform Flow}
Uniform flow is precisely that, a flow that uniformly moves in a single direction without variation in the flow field.
Mathematically, we describe the potential of a uniform flow as\todo{redo all these margin plots in pure tikz...}

\begin{marginfigure}
	\begin{tikzpicture}[/tikz/background rectangle/.style={fill={rgb,1:red,1.0;green,1.0;blue,1.0}, draw opacity={1.0}}, show background rectangle]
\begin{axis}[point meta max={nan}, point meta min={nan}, title={}, title style={at={{(0.5,1)}}, anchor={south}, font={{\fontsize{14 pt}{18.2 pt}\selectfont}}, color={rgb,1:red,0.0;green,0.0;blue,0.0}, draw opacity={1.0}, rotate={0.0}}, legend style={color={rgb,1:red,0.0;green,0.0;blue,0.0}, draw opacity={0.0}, line width={1}, solid, fill={rgb,1:red,0.0;green,0.0;blue,0.0}, fill opacity={0.0}, text opacity={1.0}, font={{\fontsize{8 pt}{10.4 pt}\selectfont}}, text={rgb,1:red,0.0;green,0.0;blue,0.0}, cells={anchor={center}}, at={(1.02, 1)}, anchor={north west}}, axis background/.style={fill={rgb,1:red,0.0;green,0.0;blue,0.0}, opacity={0.0}}, anchor={north west}, xshift={1.0mm}, yshift={-1.0mm}, width={48.8mm}, height={48.8mm}, scaled x ticks={false}, xlabel={}, x tick style={color={rgb,1:red,0.0;green,0.0;blue,0.0}, opacity={1.0}}, x tick label style={color={rgb,1:red,0.0;green,0.0;blue,0.0}, opacity={1.0}, rotate={0}}, xlabel style={at={(ticklabel cs:0.5)}, anchor=near ticklabel, font={{\fontsize{11 pt}{14.3 pt}\selectfont}}, color={rgb,1:red,0.0;green,0.0;blue,0.0}, draw opacity={1.0}, rotate={0.0}}, xmajorticks={false}, xmajorgrids={false}, xmin={0}, xmax={2}, axis x line*={left}, separate axis lines, x axis line style={{draw opacity = 0}}, scaled y ticks={false}, ylabel={}, y tick style={color={rgb,1:red,0.0;green,0.0;blue,0.0}, opacity={1.0}}, y tick label style={color={rgb,1:red,0.0;green,0.0;blue,0.0}, opacity={1.0}, rotate={0}}, ylabel style={at={(ticklabel cs:0.5)}, anchor=near ticklabel, font={{\fontsize{11 pt}{14.3 pt}\selectfont}}, color={rgb,1:red,0.0;green,0.0;blue,0.0}, draw opacity={1.0}, rotate={0.0}}, ymajorticks={false}, ymajorgrids={false}, ymin={0}, ymax={3}, axis y line*={left}, y axis line style={{draw opacity = 0}}, colorbar={false}]
    \addplot[color={rgb,1:red,0.0;green,0.1804;blue,0.3647}, name path={3f23e095-9dc1-462b-ad9f-feadfaa64d14}, draw opacity={1.0}, line width={2.0}, solid, quiver={u={\thisrow{u}}, v={\thisrow{v}}, every arrow/.append style={-{stealth[length = 0.3pt, width = 0.3pt]}}}]
        table[row sep={\\}]
        {
            x  y  u  v  \\
            0.0  0.05  2.0  0.2  \\
        }
        ;
    \addplot[color={rgb,1:red,0.0;green,0.1804;blue,0.3647}, name path={3f23e095-9dc1-462b-ad9f-feadfaa64d14}, draw opacity={1.0}, line width={2.0}, solid]
        table[row sep={\\}]
        {
            \\
            0.0  0.05  \\
        }
        ;
    \addplot[color={rgb,1:red,0.0;green,0.1804;blue,0.3647}, name path={9d7e09b5-7199-4b12-acbc-c4e646394a7b}, draw opacity={1.0}, line width={2.0}, solid, quiver={u={\thisrow{u}}, v={\thisrow{v}}, every arrow/.append style={-{stealth[length = 0.3pt, width = 0.3pt]}}}]
        table[row sep={\\}]
        {
            x  y  u  v  \\
            0.0  0.39999999999999997  2.0  0.2  \\
        }
        ;
    \addplot[color={rgb,1:red,0.0;green,0.1804;blue,0.3647}, name path={9d7e09b5-7199-4b12-acbc-c4e646394a7b}, draw opacity={1.0}, line width={2.0}, solid]
        table[row sep={\\}]
        {
            \\
            0.0  0.39999999999999997  \\
        }
        ;
    \addplot[color={rgb,1:red,0.0;green,0.1804;blue,0.3647}, name path={4b9a0d21-1d52-47d8-8ff5-02096d9d77c1}, draw opacity={1.0}, line width={2.0}, solid, quiver={u={\thisrow{u}}, v={\thisrow{v}}, every arrow/.append style={-{stealth[length = 0.3pt, width = 0.3pt]}}}]
        table[row sep={\\}]
        {
            x  y  u  v  \\
            0.0  0.75  2.0  0.19999999999999996  \\
        }
        ;
    \addplot[color={rgb,1:red,0.0;green,0.1804;blue,0.3647}, name path={4b9a0d21-1d52-47d8-8ff5-02096d9d77c1}, draw opacity={1.0}, line width={2.0}, solid]
        table[row sep={\\}]
        {
            \\
            0.0  0.75  \\
        }
        ;
    \addplot[color={rgb,1:red,0.0;green,0.1804;blue,0.3647}, name path={6be39b1e-c5ff-4dd5-9fee-e5750030fdab}, draw opacity={1.0}, line width={2.0}, solid, quiver={u={\thisrow{u}}, v={\thisrow{v}}, every arrow/.append style={-{stealth[length = 0.3pt, width = 0.3pt]}}}]
        table[row sep={\\}]
        {
            x  y  u  v  \\
            0.0  1.0999999999999999  2.0  0.19999999999999996  \\
        }
        ;
    \addplot[color={rgb,1:red,0.0;green,0.1804;blue,0.3647}, name path={6be39b1e-c5ff-4dd5-9fee-e5750030fdab}, draw opacity={1.0}, line width={2.0}, solid]
        table[row sep={\\}]
        {
            \\
            0.0  1.0999999999999999  \\
        }
        ;
    \addplot[color={rgb,1:red,0.0;green,0.1804;blue,0.3647}, name path={beea7ddb-9c5d-4bef-b34d-bece73643fad}, draw opacity={1.0}, line width={2.0}, solid, quiver={u={\thisrow{u}}, v={\thisrow{v}}, every arrow/.append style={-{stealth[length = 0.3pt, width = 0.3pt]}}}]
        table[row sep={\\}]
        {
            x  y  u  v  \\
            0.0  1.45  2.0  0.19999999999999996  \\
        }
        ;
    \addplot[color={rgb,1:red,0.0;green,0.1804;blue,0.3647}, name path={beea7ddb-9c5d-4bef-b34d-bece73643fad}, draw opacity={1.0}, line width={2.0}, solid]
        table[row sep={\\}]
        {
            \\
            0.0  1.45  \\
        }
        ;
    \addplot[color={rgb,1:red,0.0;green,0.1804;blue,0.3647}, name path={2b1674ee-02b0-43f5-a89f-4f5719949cb2}, draw opacity={1.0}, line width={2.0}, solid, quiver={u={\thisrow{u}}, v={\thisrow{v}}, every arrow/.append style={-{stealth[length = 0.3pt, width = 0.3pt]}}}]
        table[row sep={\\}]
        {
            x  y  u  v  \\
            0.0  1.8  2.0  0.19999999999999996  \\
        }
        ;
    \addplot[color={rgb,1:red,0.0;green,0.1804;blue,0.3647}, name path={2b1674ee-02b0-43f5-a89f-4f5719949cb2}, draw opacity={1.0}, line width={2.0}, solid]
        table[row sep={\\}]
        {
            \\
            0.0  1.8  \\
        }
        ;
    \addplot[color={rgb,1:red,0.0;green,0.1804;blue,0.3647}, name path={a2aa0364-2fc6-4eba-b130-31657c14e45c}, draw opacity={1.0}, line width={2.0}, solid, quiver={u={\thisrow{u}}, v={\thisrow{v}}, every arrow/.append style={-{stealth[length = 0.3pt, width = 0.3pt]}}}]
        table[row sep={\\}]
        {
            x  y  u  v  \\
            0.0  2.1499999999999995  2.0  0.20000000000000018  \\
        }
        ;
    \addplot[color={rgb,1:red,0.0;green,0.1804;blue,0.3647}, name path={a2aa0364-2fc6-4eba-b130-31657c14e45c}, draw opacity={1.0}, line width={2.0}, solid]
        table[row sep={\\}]
        {
            \\
            0.0  2.1499999999999995  \\
        }
        ;
    \addplot[color={rgb,1:red,0.0;green,0.1804;blue,0.3647}, name path={aa6232fb-62b5-41a8-8e6c-9328f1a23d23}, draw opacity={1.0}, line width={2.0}, solid, quiver={u={\thisrow{u}}, v={\thisrow{v}}, every arrow/.append style={-{stealth[length = 0.3pt, width = 0.3pt]}}}]
        table[row sep={\\}]
        {
            x  y  u  v  \\
            0.0  2.4999999999999996  2.0  0.20000000000000018  \\
        }
        ;
    \addplot[color={rgb,1:red,0.0;green,0.1804;blue,0.3647}, name path={aa6232fb-62b5-41a8-8e6c-9328f1a23d23}, draw opacity={1.0}, line width={2.0}, solid]
        table[row sep={\\}]
        {
            \\
            0.0  2.4999999999999996  \\
        }
        ;
\end{axis}
\end{tikzpicture}

\end{marginfigure}

\begin{equation}
\label{eqn:uniformflow}
	\phi_{u} = V_\infty \hat{r}
\end{equation}

\noindent where \(V_\infty\) is the magnitude of the flow and \(\hat{r}\) in a vector indicating the direction of the flow.


\subsubsection{Source/Sink Flow}

\begin{marginfigure}
	\begin{tikzpicture}[/tikz/background rectangle/.style={fill={rgb,1:red,1.0;green,1.0;blue,1.0}, draw opacity={1.0}}, show background rectangle]
\begin{axis}[point meta max={nan}, point meta min={nan}, title={}, title style={at={{(0.5,1)}}, anchor={south}, font={{\fontsize{14 pt}{18.2 pt}\selectfont}}, color={rgb,1:red,0.0;green,0.0;blue,0.0}, draw opacity={1.0}, rotate={0.0}}, legend style={color={rgb,1:red,0.0;green,0.0;blue,0.0}, draw opacity={0.0}, line width={1}, solid, fill={rgb,1:red,0.0;green,0.0;blue,0.0}, fill opacity={0.0}, text opacity={1.0}, font={{\fontsize{8 pt}{10.4 pt}\selectfont}}, text={rgb,1:red,0.0;green,0.0;blue,0.0}, cells={anchor={center}}, at={(1.02, 1)}, anchor={north west}}, axis background/.style={fill={rgb,1:red,0.0;green,0.0;blue,0.0}, opacity={0.0}}, anchor={north west}, xshift={1.0mm}, yshift={-1.0mm}, width={48.8mm}, height={48.8mm}, scaled x ticks={false}, xlabel={}, x tick style={color={rgb,1:red,0.0;green,0.0;blue,0.0}, opacity={1.0}}, x tick label style={color={rgb,1:red,0.0;green,0.0;blue,0.0}, opacity={1.0}, rotate={0}}, xlabel style={at={(ticklabel cs:0.5)}, anchor=near ticklabel, font={{\fontsize{11 pt}{14.3 pt}\selectfont}}, color={rgb,1:red,0.0;green,0.0;blue,0.0}, draw opacity={1.0}, rotate={0.0}}, xmajorticks={false}, xmajorgrids={false}, xmin={0}, xmax={3}, axis x line*={left}, separate axis lines, x axis line style={{draw opacity = 0}}, scaled y ticks={false}, ylabel={}, y tick style={color={rgb,1:red,0.0;green,0.0;blue,0.0}, opacity={1.0}}, y tick label style={color={rgb,1:red,0.0;green,0.0;blue,0.0}, opacity={1.0}, rotate={0}}, ylabel style={at={(ticklabel cs:0.5)}, anchor=near ticklabel, font={{\fontsize{11 pt}{14.3 pt}\selectfont}}, color={rgb,1:red,0.0;green,0.0;blue,0.0}, draw opacity={1.0}, rotate={0.0}}, ymajorticks={false}, ymajorgrids={false}, ymin={0}, ymax={3}, axis y line*={left}, y axis line style={{draw opacity = 0}}, colorbar={false}]
    \addplot[color={rgb,1:red,0.0;green,0.1804;blue,0.3647}, name path={ace2c930-df03-40dc-ac4b-26ed23eff283}, draw opacity={1.0}, line width={2.0}, solid, mark={*}, mark size={3.0 pt}, mark repeat={1}, mark options={color={rgb,1:red,0.0;green,0.0;blue,0.0}, draw opacity={0.0}, fill={rgb,1:red,0.0;green,0.1804;blue,0.3647}, fill opacity={1.0}, line width={0.75}, rotate={0}, solid}]
        table[row sep={\\}]
        {
            \\
            1.5  1.5  \\
        }
        ;
    \addplot[color={rgb,1:red,0.0;green,0.1804;blue,0.3647}, name path={02b55788-007d-49d8-a98c-1195d725f92c}, draw opacity={1.0}, line width={2.0}, solid, quiver={u={\thisrow{u}}, v={\thisrow{v}}, every arrow/.append style={-{stealth[length = 0.3pt, width = 0.3pt]}}}]
        table[row sep={\\}]
        {
            x  y  u  v  \\
            1.5  1.5  0.0  1.5  \\
        }
        ;
    \addplot[color={rgb,1:red,0.0;green,0.1804;blue,0.3647}, name path={02b55788-007d-49d8-a98c-1195d725f92c}, draw opacity={1.0}, line width={2.0}, solid]
        table[row sep={\\}]
        {
            \\
            1.5  1.5  \\
        }
        ;
    \addplot[color={rgb,1:red,0.0;green,0.1804;blue,0.3647}, name path={cbdeeec8-9cbf-4690-a482-cec077e2821d}, draw opacity={1.0}, line width={2.0}, solid, quiver={u={\thisrow{u}}, v={\thisrow{v}}, every arrow/.append style={-{stealth[length = 0.3pt, width = 0.3pt]}}}]
        table[row sep={\\}]
        {
            x  y  u  v  \\
            1.5  1.5  -0.8109612261833965  1.261880299246772  \\
        }
        ;
    \addplot[color={rgb,1:red,0.0;green,0.1804;blue,0.3647}, name path={cbdeeec8-9cbf-4690-a482-cec077e2821d}, draw opacity={1.0}, line width={2.0}, solid]
        table[row sep={\\}]
        {
            \\
            1.5  1.5  \\
        }
        ;
    \addplot[color={rgb,1:red,0.0;green,0.1804;blue,0.3647}, name path={8ba7b7cc-2dc8-4852-a80e-926678a07f42}, draw opacity={1.0}, line width={2.0}, solid, quiver={u={\thisrow{u}}, v={\thisrow{v}}, every arrow/.append style={-{stealth[length = 0.3pt, width = 0.3pt]}}}]
        table[row sep={\\}]
        {
            x  y  u  v  \\
            1.5  1.5  -1.3644479930317774  0.62312251950283  \\
        }
        ;
    \addplot[color={rgb,1:red,0.0;green,0.1804;blue,0.3647}, name path={8ba7b7cc-2dc8-4852-a80e-926678a07f42}, draw opacity={1.0}, line width={2.0}, solid]
        table[row sep={\\}]
        {
            \\
            1.5  1.5  \\
        }
        ;
    \addplot[color={rgb,1:red,0.0;green,0.1804;blue,0.3647}, name path={11d91c0d-e687-4f09-9e9a-b0c3e5bbd641}, draw opacity={1.0}, line width={2.0}, solid, quiver={u={\thisrow{u}}, v={\thisrow{v}}, every arrow/.append style={-{stealth[length = 0.3pt, width = 0.3pt]}}}]
        table[row sep={\\}]
        {
            x  y  u  v  \\
            1.5  1.5  -1.4847321628213992  -0.21347225740992748  \\
        }
        ;
    \addplot[color={rgb,1:red,0.0;green,0.1804;blue,0.3647}, name path={11d91c0d-e687-4f09-9e9a-b0c3e5bbd641}, draw opacity={1.0}, line width={2.0}, solid]
        table[row sep={\\}]
        {
            \\
            1.5  1.5  \\
        }
        ;
    \addplot[color={rgb,1:red,0.0;green,0.1804;blue,0.3647}, name path={4511db2a-4f06-41ff-a42b-a6c01951648f}, draw opacity={1.0}, line width={2.0}, solid, quiver={u={\thisrow{u}}, v={\thisrow{v}}, every arrow/.append style={-{stealth[length = 0.3pt, width = 0.3pt]}}}]
        table[row sep={\\}]
        {
            x  y  u  v  \\
            1.5  1.5  -1.1336243615313877  -0.9822911009179275  \\
        }
        ;
    \addplot[color={rgb,1:red,0.0;green,0.1804;blue,0.3647}, name path={4511db2a-4f06-41ff-a42b-a6c01951648f}, draw opacity={1.0}, line width={2.0}, solid]
        table[row sep={\\}]
        {
            \\
            1.5  1.5  \\
        }
        ;
    \addplot[color={rgb,1:red,0.0;green,0.1804;blue,0.3647}, name path={f8c10f12-9052-409a-956f-1b43d3ba6231}, draw opacity={1.0}, line width={2.0}, solid, quiver={u={\thisrow{u}}, v={\thisrow{v}}, every arrow/.append style={-{stealth[length = 0.3pt, width = 0.3pt]}}}]
        table[row sep={\\}]
        {
            x  y  u  v  \\
            1.5  1.5  -0.4225988352621446  -1.439239460421746  \\
        }
        ;
    \addplot[color={rgb,1:red,0.0;green,0.1804;blue,0.3647}, name path={f8c10f12-9052-409a-956f-1b43d3ba6231}, draw opacity={1.0}, line width={2.0}, solid]
        table[row sep={\\}]
        {
            \\
            1.5  1.5  \\
        }
        ;
    \addplot[color={rgb,1:red,0.0;green,0.1804;blue,0.3647}, name path={03685822-702a-4f48-b8f4-4e08ef26c575}, draw opacity={1.0}, line width={2.0}, solid, quiver={u={\thisrow{u}}, v={\thisrow{v}}, every arrow/.append style={-{stealth[length = 0.3pt, width = 0.3pt]}}}]
        table[row sep={\\}]
        {
            x  y  u  v  \\
            1.5  1.5  0.42259883526214415  -1.4392394604217462  \\
        }
        ;
    \addplot[color={rgb,1:red,0.0;green,0.1804;blue,0.3647}, name path={03685822-702a-4f48-b8f4-4e08ef26c575}, draw opacity={1.0}, line width={2.0}, solid]
        table[row sep={\\}]
        {
            \\
            1.5  1.5  \\
        }
        ;
    \addplot[color={rgb,1:red,0.0;green,0.1804;blue,0.3647}, name path={9ca1135c-e9a7-4f2a-9ccc-dfcd74d78c37}, draw opacity={1.0}, line width={2.0}, solid, quiver={u={\thisrow{u}}, v={\thisrow{v}}, every arrow/.append style={-{stealth[length = 0.3pt, width = 0.3pt]}}}]
        table[row sep={\\}]
        {
            x  y  u  v  \\
            1.5  1.5  1.1336243615313877  -0.9822911009179274  \\
        }
        ;
    \addplot[color={rgb,1:red,0.0;green,0.1804;blue,0.3647}, name path={9ca1135c-e9a7-4f2a-9ccc-dfcd74d78c37}, draw opacity={1.0}, line width={2.0}, solid]
        table[row sep={\\}]
        {
            \\
            1.5  1.5  \\
        }
        ;
    \addplot[color={rgb,1:red,0.0;green,0.1804;blue,0.3647}, name path={6fbe4b0b-ffbf-4f43-ade2-4ea20db7b71e}, draw opacity={1.0}, line width={2.0}, solid, quiver={u={\thisrow{u}}, v={\thisrow{v}}, every arrow/.append style={-{stealth[length = 0.3pt, width = 0.3pt]}}}]
        table[row sep={\\}]
        {
            x  y  u  v  \\
            1.5  1.5  1.4847321628213992  -0.21347225740992792  \\
        }
        ;
    \addplot[color={rgb,1:red,0.0;green,0.1804;blue,0.3647}, name path={6fbe4b0b-ffbf-4f43-ade2-4ea20db7b71e}, draw opacity={1.0}, line width={2.0}, solid]
        table[row sep={\\}]
        {
            \\
            1.5  1.5  \\
        }
        ;
    \addplot[color={rgb,1:red,0.0;green,0.1804;blue,0.3647}, name path={8324512e-d809-4463-8377-a2df8d74ca3f}, draw opacity={1.0}, line width={2.0}, solid, quiver={u={\thisrow{u}}, v={\thisrow{v}}, every arrow/.append style={-{stealth[length = 0.3pt, width = 0.3pt]}}}]
        table[row sep={\\}]
        {
            x  y  u  v  \\
            1.5  1.5  1.3644479930317779  0.6231225195028287  \\
        }
        ;
    \addplot[color={rgb,1:red,0.0;green,0.1804;blue,0.3647}, name path={8324512e-d809-4463-8377-a2df8d74ca3f}, draw opacity={1.0}, line width={2.0}, solid]
        table[row sep={\\}]
        {
            \\
            1.5  1.5  \\
        }
        ;
    \addplot[color={rgb,1:red,0.0;green,0.1804;blue,0.3647}, name path={ad826bcd-4630-48ac-b1b7-1e1d31cf8667}, draw opacity={1.0}, line width={2.0}, solid, quiver={u={\thisrow{u}}, v={\thisrow{v}}, every arrow/.append style={-{stealth[length = 0.3pt, width = 0.3pt]}}}]
        table[row sep={\\}]
        {
            x  y  u  v  \\
            1.5  1.5  0.8109612261833963  1.261880299246772  \\
        }
        ;
    \addplot[color={rgb,1:red,0.0;green,0.1804;blue,0.3647}, name path={ad826bcd-4630-48ac-b1b7-1e1d31cf8667}, draw opacity={1.0}, line width={2.0}, solid]
        table[row sep={\\}]
        {
            \\
            1.5  1.5  \\
        }
        ;
    \addplot[color={rgb,1:red,0.0;green,0.1804;blue,0.3647}, name path={7d48dd83-135d-45cd-9b8e-921851b77182}, draw opacity={1.0}, line width={2.0}, solid, quiver={u={\thisrow{u}}, v={\thisrow{v}}, every arrow/.append style={-{stealth[length = 0.3pt, width = 0.3pt]}}}]
        table[row sep={\\}]
        {
            x  y  u  v  \\
            1.5  1.5  4.440892098500626e-16  1.5  \\
        }
        ;
    \addplot[color={rgb,1:red,0.0;green,0.1804;blue,0.3647}, name path={7d48dd83-135d-45cd-9b8e-921851b77182}, draw opacity={1.0}, line width={2.0}, solid]
        table[row sep={\\}]
        {
            \\
            1.5  1.5  \\
        }
        ;
\end{axis}
\end{tikzpicture}

\end{marginfigure}

Source and sink flows are mathematically identical, with the exception of sign.
A source can be described as a point \textit{from} which flow extends, and a sink is the opposite, a point \textit{to} which flow extends.
The defining characteristics of these flows are that they have only radial, and no tangential, components.
Expressed mathematically, the velocity potential for source/sink flow is

\begin{marginfigure}
	\begin{tikzpicture}[/tikz/background rectangle/.style={fill={rgb,1:red,1.0;green,1.0;blue,1.0}, draw opacity={1.0}}, show background rectangle]
\begin{axis}[point meta max={nan}, point meta min={nan}, title={}, title style={at={{(0.5,1)}}, anchor={south}, font={{\fontsize{14 pt}{18.2 pt}\selectfont}}, color={rgb,1:red,0.0;green,0.0;blue,0.0}, draw opacity={1.0}, rotate={0.0}}, legend style={color={rgb,1:red,0.0;green,0.0;blue,0.0}, draw opacity={0.0}, line width={1}, solid, fill={rgb,1:red,0.0;green,0.0;blue,0.0}, fill opacity={0.0}, text opacity={1.0}, font={{\fontsize{8 pt}{10.4 pt}\selectfont}}, text={rgb,1:red,0.0;green,0.0;blue,0.0}, cells={anchor={center}}, at={(1.02, 1)}, anchor={north west}}, axis background/.style={fill={rgb,1:red,0.0;green,0.0;blue,0.0}, opacity={0.0}}, anchor={north west}, xshift={1.0mm}, yshift={-1.0mm}, width={48.8mm}, height={48.8mm}, scaled x ticks={false}, xlabel={}, x tick style={color={rgb,1:red,0.0;green,0.0;blue,0.0}, opacity={1.0}}, x tick label style={color={rgb,1:red,0.0;green,0.0;blue,0.0}, opacity={1.0}, rotate={0}}, xlabel style={at={(ticklabel cs:0.5)}, anchor=near ticklabel, font={{\fontsize{11 pt}{14.3 pt}\selectfont}}, color={rgb,1:red,0.0;green,0.0;blue,0.0}, draw opacity={1.0}, rotate={0.0}}, xmajorticks={false}, xmajorgrids={false}, xmin={0}, xmax={3}, axis x line*={left}, separate axis lines, x axis line style={{draw opacity = 0}}, scaled y ticks={false}, ylabel={}, y tick style={color={rgb,1:red,0.0;green,0.0;blue,0.0}, opacity={1.0}}, y tick label style={color={rgb,1:red,0.0;green,0.0;blue,0.0}, opacity={1.0}, rotate={0}}, ylabel style={at={(ticklabel cs:0.5)}, anchor=near ticklabel, font={{\fontsize{11 pt}{14.3 pt}\selectfont}}, color={rgb,1:red,0.0;green,0.0;blue,0.0}, draw opacity={1.0}, rotate={0.0}}, ymajorticks={false}, ymajorgrids={false}, ymin={0}, ymax={3}, axis y line*={left}, y axis line style={{draw opacity = 0}}, colorbar={false}]
    \addplot[color={rgb,1:red,0.0;green,0.1804;blue,0.3647}, name path={3a41e08d-3a87-4507-ac74-68adcccf1878}, draw opacity={1.0}, line width={2.0}, solid, mark={*}, mark size={3.0 pt}, mark repeat={1}, mark options={color={rgb,1:red,0.0;green,0.0;blue,0.0}, draw opacity={0.0}, fill={rgb,1:red,0.0;green,0.1804;blue,0.3647}, fill opacity={1.0}, line width={0.75}, rotate={0}, solid}]
        table[row sep={\\}]
        {
            \\
            1.5  1.5  \\
        }
        ;
    \addplot[color={rgb,1:red,0.0;green,0.1804;blue,0.3647}, name path={38a61fcb-6078-4a64-af08-06edda8092dc}, draw opacity={1.0}, line width={2.0}, solid, quiver={u={\thisrow{u}}, v={\thisrow{v}}, every arrow/.append style={-{stealth[length = 0.3pt, width = 0.3pt]}}}]
        table[row sep={\\}]
        {
            x  y  u  v  \\
            1.5  3.0  0.0  -1.15  \\
        }
        ;
    \addplot[color={rgb,1:red,0.0;green,0.1804;blue,0.3647}, name path={38a61fcb-6078-4a64-af08-06edda8092dc}, draw opacity={1.0}, line width={2.0}, solid]
        table[row sep={\\}]
        {
            \\
            1.5  3.0  \\
        }
        ;
    \addplot[color={rgb,1:red,0.0;green,0.1804;blue,0.3647}, name path={dd839118-715d-45dc-bb63-bdcb93bc175a}, draw opacity={1.0}, line width={2.0}, solid, quiver={u={\thisrow{u}}, v={\thisrow{v}}, every arrow/.append style={-{stealth[length = 0.3pt, width = 0.3pt]}}}]
        table[row sep={\\}]
        {
            x  y  u  v  \\
            0.6890387738166035  2.761880299246772  0.6217369400739374  -0.9674415627558586  \\
        }
        ;
    \addplot[color={rgb,1:red,0.0;green,0.1804;blue,0.3647}, name path={dd839118-715d-45dc-bb63-bdcb93bc175a}, draw opacity={1.0}, line width={2.0}, solid]
        table[row sep={\\}]
        {
            \\
            0.6890387738166035  2.761880299246772  \\
        }
        ;
    \addplot[color={rgb,1:red,0.0;green,0.1804;blue,0.3647}, name path={a52b0485-c468-479d-a84a-68a4f434a580}, draw opacity={1.0}, line width={2.0}, solid, quiver={u={\thisrow{u}}, v={\thisrow{v}}, every arrow/.append style={-{stealth[length = 0.3pt, width = 0.3pt]}}}]
        table[row sep={\\}]
        {
            x  y  u  v  \\
            0.13555200696822256  2.12312251950283  1.046076794657696  -0.4777272649521698  \\
        }
        ;
    \addplot[color={rgb,1:red,0.0;green,0.1804;blue,0.3647}, name path={a52b0485-c468-479d-a84a-68a4f434a580}, draw opacity={1.0}, line width={2.0}, solid]
        table[row sep={\\}]
        {
            \\
            0.13555200696822256  2.12312251950283  \\
        }
        ;
    \addplot[color={rgb,1:red,0.0;green,0.1804;blue,0.3647}, name path={27c4545e-7067-4cb9-988c-15a2fc2c06a0}, draw opacity={1.0}, line width={2.0}, solid, quiver={u={\thisrow{u}}, v={\thisrow{v}}, every arrow/.append style={-{stealth[length = 0.3pt, width = 0.3pt]}}}]
        table[row sep={\\}]
        {
            x  y  u  v  \\
            0.015267837178600807  1.2865277425900725  1.1382946581630726  0.16366206401427785  \\
        }
        ;
    \addplot[color={rgb,1:red,0.0;green,0.1804;blue,0.3647}, name path={27c4545e-7067-4cb9-988c-15a2fc2c06a0}, draw opacity={1.0}, line width={2.0}, solid]
        table[row sep={\\}]
        {
            \\
            0.015267837178600807  1.2865277425900725  \\
        }
        ;
    \addplot[color={rgb,1:red,0.0;green,0.1804;blue,0.3647}, name path={f3b4b8d9-7eeb-4ef1-88c9-f3184d0a19de}, draw opacity={1.0}, line width={2.0}, solid, quiver={u={\thisrow{u}}, v={\thisrow{v}}, every arrow/.append style={-{stealth[length = 0.3pt, width = 0.3pt]}}}]
        table[row sep={\\}]
        {
            x  y  u  v  \\
            0.3663756384686123  0.5177088990820725  0.8691120105073973  0.7530898440370778  \\
        }
        ;
    \addplot[color={rgb,1:red,0.0;green,0.1804;blue,0.3647}, name path={f3b4b8d9-7eeb-4ef1-88c9-f3184d0a19de}, draw opacity={1.0}, line width={2.0}, solid]
        table[row sep={\\}]
        {
            \\
            0.3663756384686123  0.5177088990820725  \\
        }
        ;
    \addplot[color={rgb,1:red,0.0;green,0.1804;blue,0.3647}, name path={0df22b81-e882-4e3e-a7d6-cf198782159f}, draw opacity={1.0}, line width={2.0}, solid, quiver={u={\thisrow{u}}, v={\thisrow{v}}, every arrow/.append style={-{stealth[length = 0.3pt, width = 0.3pt]}}}]
        table[row sep={\\}]
        {
            x  y  u  v  \\
            1.0774011647378554  0.060760539578254  0.3239924403676442  1.1034169196566719  \\
        }
        ;
    \addplot[color={rgb,1:red,0.0;green,0.1804;blue,0.3647}, name path={0df22b81-e882-4e3e-a7d6-cf198782159f}, draw opacity={1.0}, line width={2.0}, solid]
        table[row sep={\\}]
        {
            \\
            1.0774011647378554  0.060760539578254  \\
        }
        ;
    \addplot[color={rgb,1:red,0.0;green,0.1804;blue,0.3647}, name path={a79f4aeb-028a-4061-b0d3-7d459614280c}, draw opacity={1.0}, line width={2.0}, solid, quiver={u={\thisrow{u}}, v={\thisrow{v}}, every arrow/.append style={-{stealth[length = 0.3pt, width = 0.3pt]}}}]
        table[row sep={\\}]
        {
            x  y  u  v  \\
            1.9225988352621441  0.06076053957825378  -0.32399244036764396  1.103416919656672  \\
        }
        ;
    \addplot[color={rgb,1:red,0.0;green,0.1804;blue,0.3647}, name path={a79f4aeb-028a-4061-b0d3-7d459614280c}, draw opacity={1.0}, line width={2.0}, solid]
        table[row sep={\\}]
        {
            \\
            1.9225988352621441  0.06076053957825378  \\
        }
        ;
    \addplot[color={rgb,1:red,0.0;green,0.1804;blue,0.3647}, name path={84e63717-435c-4cc8-94e7-f92aec096d27}, draw opacity={1.0}, line width={2.0}, solid, quiver={u={\thisrow{u}}, v={\thisrow{v}}, every arrow/.append style={-{stealth[length = 0.3pt, width = 0.3pt]}}}]
        table[row sep={\\}]
        {
            x  y  u  v  \\
            2.6336243615313877  0.5177088990820726  -0.8691120105073973  0.7530898440370777  \\
        }
        ;
    \addplot[color={rgb,1:red,0.0;green,0.1804;blue,0.3647}, name path={84e63717-435c-4cc8-94e7-f92aec096d27}, draw opacity={1.0}, line width={2.0}, solid]
        table[row sep={\\}]
        {
            \\
            2.6336243615313877  0.5177088990820726  \\
        }
        ;
    \addplot[color={rgb,1:red,0.0;green,0.1804;blue,0.3647}, name path={6f7a49a3-fc7c-4425-9b84-b5916cbf8f71}, draw opacity={1.0}, line width={2.0}, solid, quiver={u={\thisrow{u}}, v={\thisrow{v}}, every arrow/.append style={-{stealth[length = 0.3pt, width = 0.3pt]}}}]
        table[row sep={\\}]
        {
            x  y  u  v  \\
            2.984732162821399  1.286527742590072  -1.1382946581630728  0.16366206401427807  \\
        }
        ;
    \addplot[color={rgb,1:red,0.0;green,0.1804;blue,0.3647}, name path={6f7a49a3-fc7c-4425-9b84-b5916cbf8f71}, draw opacity={1.0}, line width={2.0}, solid]
        table[row sep={\\}]
        {
            \\
            2.984732162821399  1.286527742590072  \\
        }
        ;
    \addplot[color={rgb,1:red,0.0;green,0.1804;blue,0.3647}, name path={869eb2a6-a725-45bf-84b0-a096065de24a}, draw opacity={1.0}, line width={2.0}, solid, quiver={u={\thisrow{u}}, v={\thisrow{v}}, every arrow/.append style={-{stealth[length = 0.3pt, width = 0.3pt]}}}]
        table[row sep={\\}]
        {
            x  y  u  v  \\
            2.864447993031778  2.1231225195028287  -1.0460767946576963  -0.47772726495216866  \\
        }
        ;
    \addplot[color={rgb,1:red,0.0;green,0.1804;blue,0.3647}, name path={869eb2a6-a725-45bf-84b0-a096065de24a}, draw opacity={1.0}, line width={2.0}, solid]
        table[row sep={\\}]
        {
            \\
            2.864447993031778  2.1231225195028287  \\
        }
        ;
    \addplot[color={rgb,1:red,0.0;green,0.1804;blue,0.3647}, name path={44509def-3198-4f66-ac6c-38914514cf98}, draw opacity={1.0}, line width={2.0}, solid, quiver={u={\thisrow{u}}, v={\thisrow{v}}, every arrow/.append style={-{stealth[length = 0.3pt, width = 0.3pt]}}}]
        table[row sep={\\}]
        {
            x  y  u  v  \\
            2.3109612261833963  2.761880299246772  -0.6217369400739372  -0.9674415627558586  \\
        }
        ;
    \addplot[color={rgb,1:red,0.0;green,0.1804;blue,0.3647}, name path={44509def-3198-4f66-ac6c-38914514cf98}, draw opacity={1.0}, line width={2.0}, solid]
        table[row sep={\\}]
        {
            \\
            2.3109612261833963  2.761880299246772  \\
        }
        ;
    \addplot[color={rgb,1:red,0.0;green,0.1804;blue,0.3647}, name path={5e5776ab-0b24-483a-9cb6-a11b85ad133c}, draw opacity={1.0}, line width={2.0}, solid, quiver={u={\thisrow{u}}, v={\thisrow{v}}, every arrow/.append style={-{stealth[length = 0.3pt, width = 0.3pt]}}}]
        table[row sep={\\}]
        {
            x  y  u  v  \\
            1.5000000000000004  3.0  -4.440892098500626e-16  -1.15  \\
        }
        ;
    \addplot[color={rgb,1:red,0.0;green,0.1804;blue,0.3647}, name path={5e5776ab-0b24-483a-9cb6-a11b85ad133c}, draw opacity={1.0}, line width={2.0}, solid]
        table[row sep={\\}]
        {
            \\
            1.5000000000000004  3.0  \\
        }
        ;
\end{axis}
\end{tikzpicture}

\end{marginfigure}


\begin{equation}
\label{eqn:sourceflow}
	\phi_s = \frac{\pm\Lambda}{2\pi} \ln(r)
\end{equation}

\noindent where \(\Lambda\) is the strength of the source/sink and \(r\) is the radial distance from the source/sink.
A positive sign indicates a source, and a negative, a sink.


\subsubsection{Doublet Flow}

\begin{marginfigure}
	\input{./panelmethodcontents/panelmethodfigures/doubletflow.tikz}
\end{marginfigure}

Taking a source and sink with equal strengths and moving them toward each other until infinitesimally close together does not cancel them out, but rather induces a twin-lobbed flow field.
Rather than treating them separately, this combination creates it's own elementary flow that we call a doublet.
The velocity potential for a doublet is described by

\begin{equation}
\label{eqn:doubletflow}
	\phi_d = \frac{\kappa}{2\pi} \frac{\cos\theta}{r}
\end{equation}

\noindent, where \(\kappa\) is the doublet strength and \(\theta\) is the angle relative to the doublet axis.

\subsubsection{Vortex Flow}

The final elementary flow that we will discuss here is the vortex.
Vortex flow characteristics are the inverse of source/sink flows in that there is no radial, only tangential components to the flow.
The velocity potential is therefore similar to that of a source/sink:

\begin{marginfigure}
	\input{./panelmethodcontents/panelmethodfigures/vortexflow.tikz}
\end{marginfigure}

\begin{equation}
	\label{eqn:vortexflow}
	\phi_v = \frac{\Gamma}{2\pi} \theta
\end{equation}

\noindent where \(\Gamma\) is the vortex strength.


\subsection{Superposition Examples}

\toadd{add in rankine oval example, and combine with non-lifting cylinder}
\subsubsection{Rankine Oval}
As a preview to next chapter, an interesting superposition of a uniform flow, a source, and a sink results in an oval shaped ``bubble'' in the flow field.

[todo: explain the maths and how it's just addition.]

[Also probably want to talk about how to get velocity or stream or something for plotting/postprocessing and show some figures.]

\toadd{add in lifting cylinder example to show how vortices are superimposed}
\subsubsection{Lifting Cylinder}
