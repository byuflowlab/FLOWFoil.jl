\section{Preliminaries}
\label{sec:potentialflowpreliminaries}

\subsection{Elementary Flows}
\label{ssec:elemflows}

As we begin this section, it is important to note that the Laplace equation is a \textit{linear} equation.
We will be taking advantage of that fact moving forward, as it allows us to describe complex flow fields through the superposition of many, much simpler flows.
The following are a handful of useful elementary flow solutions.

\subsubsection{Uniform Flow}
Uniform flow is precisely that, a flow that uniformly moves in a single direction without variation in the flow field.
Mathematically, we describe the potential of a uniform flow as\todo{redo all these margin plots in pure tikz...}

\begin{marginfigure}
	\begin{tikzpicture}[/tikz/background rectangle/.style={fill={rgb,1:red,1.0;green,1.0;blue,1.0}, draw opacity={1.0}}, show background rectangle]
\begin{axis}[point meta max={nan}, point meta min={nan}, title={}, title style={at={{(0.5,1)}}, anchor={south}, font={{\fontsize{14 pt}{18.2 pt}\selectfont}}, color={rgb,1:red,0.0;green,0.0;blue,0.0}, draw opacity={1.0}, rotate={0.0}}, legend style={color={rgb,1:red,0.0;green,0.0;blue,0.0}, draw opacity={0.0}, line width={1}, solid, fill={rgb,1:red,0.0;green,0.0;blue,0.0}, fill opacity={0.0}, text opacity={1.0}, font={{\fontsize{8 pt}{10.4 pt}\selectfont}}, text={rgb,1:red,0.0;green,0.0;blue,0.0}, cells={anchor={center}}, at={(1.02, 1)}, anchor={north west}}, axis background/.style={fill={rgb,1:red,0.0;green,0.0;blue,0.0}, opacity={0.0}}, anchor={north west}, xshift={1.0mm}, yshift={-1.0mm}, width={48.8mm}, height={48.8mm}, scaled x ticks={false}, xlabel={}, x tick style={color={rgb,1:red,0.0;green,0.0;blue,0.0}, opacity={1.0}}, x tick label style={color={rgb,1:red,0.0;green,0.0;blue,0.0}, opacity={1.0}, rotate={0}}, xlabel style={at={(ticklabel cs:0.5)}, anchor=near ticklabel, font={{\fontsize{11 pt}{14.3 pt}\selectfont}}, color={rgb,1:red,0.0;green,0.0;blue,0.0}, draw opacity={1.0}, rotate={0.0}}, xmajorticks={false}, xmajorgrids={false}, xmin={0}, xmax={2}, axis x line*={left}, separate axis lines, x axis line style={{draw opacity = 0}}, scaled y ticks={false}, ylabel={}, y tick style={color={rgb,1:red,0.0;green,0.0;blue,0.0}, opacity={1.0}}, y tick label style={color={rgb,1:red,0.0;green,0.0;blue,0.0}, opacity={1.0}, rotate={0}}, ylabel style={at={(ticklabel cs:0.5)}, anchor=near ticklabel, font={{\fontsize{11 pt}{14.3 pt}\selectfont}}, color={rgb,1:red,0.0;green,0.0;blue,0.0}, draw opacity={1.0}, rotate={0.0}}, ymajorticks={false}, ymajorgrids={false}, ymin={0}, ymax={3}, axis y line*={left}, y axis line style={{draw opacity = 0}}, colorbar={false}]
    \addplot[color={rgb,1:red,0.0;green,0.1804;blue,0.3647}, name path={3f23e095-9dc1-462b-ad9f-feadfaa64d14}, draw opacity={1.0}, line width={2.0}, solid, quiver={u={\thisrow{u}}, v={\thisrow{v}}, every arrow/.append style={-{stealth[length = 0.3pt, width = 0.3pt]}}}]
        table[row sep={\\}]
        {
            x  y  u  v  \\
            0.0  0.05  2.0  0.2  \\
        }
        ;
    \addplot[color={rgb,1:red,0.0;green,0.1804;blue,0.3647}, name path={3f23e095-9dc1-462b-ad9f-feadfaa64d14}, draw opacity={1.0}, line width={2.0}, solid]
        table[row sep={\\}]
        {
            \\
            0.0  0.05  \\
        }
        ;
    \addplot[color={rgb,1:red,0.0;green,0.1804;blue,0.3647}, name path={9d7e09b5-7199-4b12-acbc-c4e646394a7b}, draw opacity={1.0}, line width={2.0}, solid, quiver={u={\thisrow{u}}, v={\thisrow{v}}, every arrow/.append style={-{stealth[length = 0.3pt, width = 0.3pt]}}}]
        table[row sep={\\}]
        {
            x  y  u  v  \\
            0.0  0.39999999999999997  2.0  0.2  \\
        }
        ;
    \addplot[color={rgb,1:red,0.0;green,0.1804;blue,0.3647}, name path={9d7e09b5-7199-4b12-acbc-c4e646394a7b}, draw opacity={1.0}, line width={2.0}, solid]
        table[row sep={\\}]
        {
            \\
            0.0  0.39999999999999997  \\
        }
        ;
    \addplot[color={rgb,1:red,0.0;green,0.1804;blue,0.3647}, name path={4b9a0d21-1d52-47d8-8ff5-02096d9d77c1}, draw opacity={1.0}, line width={2.0}, solid, quiver={u={\thisrow{u}}, v={\thisrow{v}}, every arrow/.append style={-{stealth[length = 0.3pt, width = 0.3pt]}}}]
        table[row sep={\\}]
        {
            x  y  u  v  \\
            0.0  0.75  2.0  0.19999999999999996  \\
        }
        ;
    \addplot[color={rgb,1:red,0.0;green,0.1804;blue,0.3647}, name path={4b9a0d21-1d52-47d8-8ff5-02096d9d77c1}, draw opacity={1.0}, line width={2.0}, solid]
        table[row sep={\\}]
        {
            \\
            0.0  0.75  \\
        }
        ;
    \addplot[color={rgb,1:red,0.0;green,0.1804;blue,0.3647}, name path={6be39b1e-c5ff-4dd5-9fee-e5750030fdab}, draw opacity={1.0}, line width={2.0}, solid, quiver={u={\thisrow{u}}, v={\thisrow{v}}, every arrow/.append style={-{stealth[length = 0.3pt, width = 0.3pt]}}}]
        table[row sep={\\}]
        {
            x  y  u  v  \\
            0.0  1.0999999999999999  2.0  0.19999999999999996  \\
        }
        ;
    \addplot[color={rgb,1:red,0.0;green,0.1804;blue,0.3647}, name path={6be39b1e-c5ff-4dd5-9fee-e5750030fdab}, draw opacity={1.0}, line width={2.0}, solid]
        table[row sep={\\}]
        {
            \\
            0.0  1.0999999999999999  \\
        }
        ;
    \addplot[color={rgb,1:red,0.0;green,0.1804;blue,0.3647}, name path={beea7ddb-9c5d-4bef-b34d-bece73643fad}, draw opacity={1.0}, line width={2.0}, solid, quiver={u={\thisrow{u}}, v={\thisrow{v}}, every arrow/.append style={-{stealth[length = 0.3pt, width = 0.3pt]}}}]
        table[row sep={\\}]
        {
            x  y  u  v  \\
            0.0  1.45  2.0  0.19999999999999996  \\
        }
        ;
    \addplot[color={rgb,1:red,0.0;green,0.1804;blue,0.3647}, name path={beea7ddb-9c5d-4bef-b34d-bece73643fad}, draw opacity={1.0}, line width={2.0}, solid]
        table[row sep={\\}]
        {
            \\
            0.0  1.45  \\
        }
        ;
    \addplot[color={rgb,1:red,0.0;green,0.1804;blue,0.3647}, name path={2b1674ee-02b0-43f5-a89f-4f5719949cb2}, draw opacity={1.0}, line width={2.0}, solid, quiver={u={\thisrow{u}}, v={\thisrow{v}}, every arrow/.append style={-{stealth[length = 0.3pt, width = 0.3pt]}}}]
        table[row sep={\\}]
        {
            x  y  u  v  \\
            0.0  1.8  2.0  0.19999999999999996  \\
        }
        ;
    \addplot[color={rgb,1:red,0.0;green,0.1804;blue,0.3647}, name path={2b1674ee-02b0-43f5-a89f-4f5719949cb2}, draw opacity={1.0}, line width={2.0}, solid]
        table[row sep={\\}]
        {
            \\
            0.0  1.8  \\
        }
        ;
    \addplot[color={rgb,1:red,0.0;green,0.1804;blue,0.3647}, name path={a2aa0364-2fc6-4eba-b130-31657c14e45c}, draw opacity={1.0}, line width={2.0}, solid, quiver={u={\thisrow{u}}, v={\thisrow{v}}, every arrow/.append style={-{stealth[length = 0.3pt, width = 0.3pt]}}}]
        table[row sep={\\}]
        {
            x  y  u  v  \\
            0.0  2.1499999999999995  2.0  0.20000000000000018  \\
        }
        ;
    \addplot[color={rgb,1:red,0.0;green,0.1804;blue,0.3647}, name path={a2aa0364-2fc6-4eba-b130-31657c14e45c}, draw opacity={1.0}, line width={2.0}, solid]
        table[row sep={\\}]
        {
            \\
            0.0  2.1499999999999995  \\
        }
        ;
    \addplot[color={rgb,1:red,0.0;green,0.1804;blue,0.3647}, name path={aa6232fb-62b5-41a8-8e6c-9328f1a23d23}, draw opacity={1.0}, line width={2.0}, solid, quiver={u={\thisrow{u}}, v={\thisrow{v}}, every arrow/.append style={-{stealth[length = 0.3pt, width = 0.3pt]}}}]
        table[row sep={\\}]
        {
            x  y  u  v  \\
            0.0  2.4999999999999996  2.0  0.20000000000000018  \\
        }
        ;
    \addplot[color={rgb,1:red,0.0;green,0.1804;blue,0.3647}, name path={aa6232fb-62b5-41a8-8e6c-9328f1a23d23}, draw opacity={1.0}, line width={2.0}, solid]
        table[row sep={\\}]
        {
            \\
            0.0  2.4999999999999996  \\
        }
        ;
\end{axis}
\end{tikzpicture}

\end{marginfigure}

\begin{equation}
\label{eqn:uniformflow}
	\phi_{u} = V_\infty \hat{r}
\end{equation}

\noindent where \(V_\infty\) is the magnitude of the flow and \(\hat{r}\) in a vector indicating the direction of the flow.


\subsubsection{Source/Sink Flow}

\begin{marginfigure}
	\begin{tikzpicture}[/tikz/background rectangle/.style={fill={rgb,1:red,1.0;green,1.0;blue,1.0}, draw opacity={1.0}}, show background rectangle]
\begin{axis}[point meta max={nan}, point meta min={nan}, title={}, title style={at={{(0.5,1)}}, anchor={south}, font={{\fontsize{14 pt}{18.2 pt}\selectfont}}, color={rgb,1:red,0.0;green,0.0;blue,0.0}, draw opacity={1.0}, rotate={0.0}}, legend style={color={rgb,1:red,0.0;green,0.0;blue,0.0}, draw opacity={0.0}, line width={1}, solid, fill={rgb,1:red,0.0;green,0.0;blue,0.0}, fill opacity={0.0}, text opacity={1.0}, font={{\fontsize{8 pt}{10.4 pt}\selectfont}}, text={rgb,1:red,0.0;green,0.0;blue,0.0}, cells={anchor={center}}, at={(1.02, 1)}, anchor={north west}}, axis background/.style={fill={rgb,1:red,0.0;green,0.0;blue,0.0}, opacity={0.0}}, anchor={north west}, xshift={1.0mm}, yshift={-1.0mm}, width={48.8mm}, height={48.8mm}, scaled x ticks={false}, xlabel={}, x tick style={color={rgb,1:red,0.0;green,0.0;blue,0.0}, opacity={1.0}}, x tick label style={color={rgb,1:red,0.0;green,0.0;blue,0.0}, opacity={1.0}, rotate={0}}, xlabel style={at={(ticklabel cs:0.5)}, anchor=near ticklabel, font={{\fontsize{11 pt}{14.3 pt}\selectfont}}, color={rgb,1:red,0.0;green,0.0;blue,0.0}, draw opacity={1.0}, rotate={0.0}}, xmajorticks={false}, xmajorgrids={false}, xmin={0}, xmax={3}, axis x line*={left}, separate axis lines, x axis line style={{draw opacity = 0}}, scaled y ticks={false}, ylabel={}, y tick style={color={rgb,1:red,0.0;green,0.0;blue,0.0}, opacity={1.0}}, y tick label style={color={rgb,1:red,0.0;green,0.0;blue,0.0}, opacity={1.0}, rotate={0}}, ylabel style={at={(ticklabel cs:0.5)}, anchor=near ticklabel, font={{\fontsize{11 pt}{14.3 pt}\selectfont}}, color={rgb,1:red,0.0;green,0.0;blue,0.0}, draw opacity={1.0}, rotate={0.0}}, ymajorticks={false}, ymajorgrids={false}, ymin={0}, ymax={3}, axis y line*={left}, y axis line style={{draw opacity = 0}}, colorbar={false}]
    \addplot[color={rgb,1:red,0.0;green,0.1804;blue,0.3647}, name path={ace2c930-df03-40dc-ac4b-26ed23eff283}, draw opacity={1.0}, line width={2.0}, solid, mark={*}, mark size={3.0 pt}, mark repeat={1}, mark options={color={rgb,1:red,0.0;green,0.0;blue,0.0}, draw opacity={0.0}, fill={rgb,1:red,0.0;green,0.1804;blue,0.3647}, fill opacity={1.0}, line width={0.75}, rotate={0}, solid}]
        table[row sep={\\}]
        {
            \\
            1.5  1.5  \\
        }
        ;
    \addplot[color={rgb,1:red,0.0;green,0.1804;blue,0.3647}, name path={02b55788-007d-49d8-a98c-1195d725f92c}, draw opacity={1.0}, line width={2.0}, solid, quiver={u={\thisrow{u}}, v={\thisrow{v}}, every arrow/.append style={-{stealth[length = 0.3pt, width = 0.3pt]}}}]
        table[row sep={\\}]
        {
            x  y  u  v  \\
            1.5  1.5  0.0  1.5  \\
        }
        ;
    \addplot[color={rgb,1:red,0.0;green,0.1804;blue,0.3647}, name path={02b55788-007d-49d8-a98c-1195d725f92c}, draw opacity={1.0}, line width={2.0}, solid]
        table[row sep={\\}]
        {
            \\
            1.5  1.5  \\
        }
        ;
    \addplot[color={rgb,1:red,0.0;green,0.1804;blue,0.3647}, name path={cbdeeec8-9cbf-4690-a482-cec077e2821d}, draw opacity={1.0}, line width={2.0}, solid, quiver={u={\thisrow{u}}, v={\thisrow{v}}, every arrow/.append style={-{stealth[length = 0.3pt, width = 0.3pt]}}}]
        table[row sep={\\}]
        {
            x  y  u  v  \\
            1.5  1.5  -0.8109612261833965  1.261880299246772  \\
        }
        ;
    \addplot[color={rgb,1:red,0.0;green,0.1804;blue,0.3647}, name path={cbdeeec8-9cbf-4690-a482-cec077e2821d}, draw opacity={1.0}, line width={2.0}, solid]
        table[row sep={\\}]
        {
            \\
            1.5  1.5  \\
        }
        ;
    \addplot[color={rgb,1:red,0.0;green,0.1804;blue,0.3647}, name path={8ba7b7cc-2dc8-4852-a80e-926678a07f42}, draw opacity={1.0}, line width={2.0}, solid, quiver={u={\thisrow{u}}, v={\thisrow{v}}, every arrow/.append style={-{stealth[length = 0.3pt, width = 0.3pt]}}}]
        table[row sep={\\}]
        {
            x  y  u  v  \\
            1.5  1.5  -1.3644479930317774  0.62312251950283  \\
        }
        ;
    \addplot[color={rgb,1:red,0.0;green,0.1804;blue,0.3647}, name path={8ba7b7cc-2dc8-4852-a80e-926678a07f42}, draw opacity={1.0}, line width={2.0}, solid]
        table[row sep={\\}]
        {
            \\
            1.5  1.5  \\
        }
        ;
    \addplot[color={rgb,1:red,0.0;green,0.1804;blue,0.3647}, name path={11d91c0d-e687-4f09-9e9a-b0c3e5bbd641}, draw opacity={1.0}, line width={2.0}, solid, quiver={u={\thisrow{u}}, v={\thisrow{v}}, every arrow/.append style={-{stealth[length = 0.3pt, width = 0.3pt]}}}]
        table[row sep={\\}]
        {
            x  y  u  v  \\
            1.5  1.5  -1.4847321628213992  -0.21347225740992748  \\
        }
        ;
    \addplot[color={rgb,1:red,0.0;green,0.1804;blue,0.3647}, name path={11d91c0d-e687-4f09-9e9a-b0c3e5bbd641}, draw opacity={1.0}, line width={2.0}, solid]
        table[row sep={\\}]
        {
            \\
            1.5  1.5  \\
        }
        ;
    \addplot[color={rgb,1:red,0.0;green,0.1804;blue,0.3647}, name path={4511db2a-4f06-41ff-a42b-a6c01951648f}, draw opacity={1.0}, line width={2.0}, solid, quiver={u={\thisrow{u}}, v={\thisrow{v}}, every arrow/.append style={-{stealth[length = 0.3pt, width = 0.3pt]}}}]
        table[row sep={\\}]
        {
            x  y  u  v  \\
            1.5  1.5  -1.1336243615313877  -0.9822911009179275  \\
        }
        ;
    \addplot[color={rgb,1:red,0.0;green,0.1804;blue,0.3647}, name path={4511db2a-4f06-41ff-a42b-a6c01951648f}, draw opacity={1.0}, line width={2.0}, solid]
        table[row sep={\\}]
        {
            \\
            1.5  1.5  \\
        }
        ;
    \addplot[color={rgb,1:red,0.0;green,0.1804;blue,0.3647}, name path={f8c10f12-9052-409a-956f-1b43d3ba6231}, draw opacity={1.0}, line width={2.0}, solid, quiver={u={\thisrow{u}}, v={\thisrow{v}}, every arrow/.append style={-{stealth[length = 0.3pt, width = 0.3pt]}}}]
        table[row sep={\\}]
        {
            x  y  u  v  \\
            1.5  1.5  -0.4225988352621446  -1.439239460421746  \\
        }
        ;
    \addplot[color={rgb,1:red,0.0;green,0.1804;blue,0.3647}, name path={f8c10f12-9052-409a-956f-1b43d3ba6231}, draw opacity={1.0}, line width={2.0}, solid]
        table[row sep={\\}]
        {
            \\
            1.5  1.5  \\
        }
        ;
    \addplot[color={rgb,1:red,0.0;green,0.1804;blue,0.3647}, name path={03685822-702a-4f48-b8f4-4e08ef26c575}, draw opacity={1.0}, line width={2.0}, solid, quiver={u={\thisrow{u}}, v={\thisrow{v}}, every arrow/.append style={-{stealth[length = 0.3pt, width = 0.3pt]}}}]
        table[row sep={\\}]
        {
            x  y  u  v  \\
            1.5  1.5  0.42259883526214415  -1.4392394604217462  \\
        }
        ;
    \addplot[color={rgb,1:red,0.0;green,0.1804;blue,0.3647}, name path={03685822-702a-4f48-b8f4-4e08ef26c575}, draw opacity={1.0}, line width={2.0}, solid]
        table[row sep={\\}]
        {
            \\
            1.5  1.5  \\
        }
        ;
    \addplot[color={rgb,1:red,0.0;green,0.1804;blue,0.3647}, name path={9ca1135c-e9a7-4f2a-9ccc-dfcd74d78c37}, draw opacity={1.0}, line width={2.0}, solid, quiver={u={\thisrow{u}}, v={\thisrow{v}}, every arrow/.append style={-{stealth[length = 0.3pt, width = 0.3pt]}}}]
        table[row sep={\\}]
        {
            x  y  u  v  \\
            1.5  1.5  1.1336243615313877  -0.9822911009179274  \\
        }
        ;
    \addplot[color={rgb,1:red,0.0;green,0.1804;blue,0.3647}, name path={9ca1135c-e9a7-4f2a-9ccc-dfcd74d78c37}, draw opacity={1.0}, line width={2.0}, solid]
        table[row sep={\\}]
        {
            \\
            1.5  1.5  \\
        }
        ;
    \addplot[color={rgb,1:red,0.0;green,0.1804;blue,0.3647}, name path={6fbe4b0b-ffbf-4f43-ade2-4ea20db7b71e}, draw opacity={1.0}, line width={2.0}, solid, quiver={u={\thisrow{u}}, v={\thisrow{v}}, every arrow/.append style={-{stealth[length = 0.3pt, width = 0.3pt]}}}]
        table[row sep={\\}]
        {
            x  y  u  v  \\
            1.5  1.5  1.4847321628213992  -0.21347225740992792  \\
        }
        ;
    \addplot[color={rgb,1:red,0.0;green,0.1804;blue,0.3647}, name path={6fbe4b0b-ffbf-4f43-ade2-4ea20db7b71e}, draw opacity={1.0}, line width={2.0}, solid]
        table[row sep={\\}]
        {
            \\
            1.5  1.5  \\
        }
        ;
    \addplot[color={rgb,1:red,0.0;green,0.1804;blue,0.3647}, name path={8324512e-d809-4463-8377-a2df8d74ca3f}, draw opacity={1.0}, line width={2.0}, solid, quiver={u={\thisrow{u}}, v={\thisrow{v}}, every arrow/.append style={-{stealth[length = 0.3pt, width = 0.3pt]}}}]
        table[row sep={\\}]
        {
            x  y  u  v  \\
            1.5  1.5  1.3644479930317779  0.6231225195028287  \\
        }
        ;
    \addplot[color={rgb,1:red,0.0;green,0.1804;blue,0.3647}, name path={8324512e-d809-4463-8377-a2df8d74ca3f}, draw opacity={1.0}, line width={2.0}, solid]
        table[row sep={\\}]
        {
            \\
            1.5  1.5  \\
        }
        ;
    \addplot[color={rgb,1:red,0.0;green,0.1804;blue,0.3647}, name path={ad826bcd-4630-48ac-b1b7-1e1d31cf8667}, draw opacity={1.0}, line width={2.0}, solid, quiver={u={\thisrow{u}}, v={\thisrow{v}}, every arrow/.append style={-{stealth[length = 0.3pt, width = 0.3pt]}}}]
        table[row sep={\\}]
        {
            x  y  u  v  \\
            1.5  1.5  0.8109612261833963  1.261880299246772  \\
        }
        ;
    \addplot[color={rgb,1:red,0.0;green,0.1804;blue,0.3647}, name path={ad826bcd-4630-48ac-b1b7-1e1d31cf8667}, draw opacity={1.0}, line width={2.0}, solid]
        table[row sep={\\}]
        {
            \\
            1.5  1.5  \\
        }
        ;
    \addplot[color={rgb,1:red,0.0;green,0.1804;blue,0.3647}, name path={7d48dd83-135d-45cd-9b8e-921851b77182}, draw opacity={1.0}, line width={2.0}, solid, quiver={u={\thisrow{u}}, v={\thisrow{v}}, every arrow/.append style={-{stealth[length = 0.3pt, width = 0.3pt]}}}]
        table[row sep={\\}]
        {
            x  y  u  v  \\
            1.5  1.5  4.440892098500626e-16  1.5  \\
        }
        ;
    \addplot[color={rgb,1:red,0.0;green,0.1804;blue,0.3647}, name path={7d48dd83-135d-45cd-9b8e-921851b77182}, draw opacity={1.0}, line width={2.0}, solid]
        table[row sep={\\}]
        {
            \\
            1.5  1.5  \\
        }
        ;
\end{axis}
\end{tikzpicture}

\end{marginfigure}

Source and sink flows are mathematically identical, with the exception of sign.
A source can be described as a point \textit{from} which flow extends, and a sink is the opposite, a point \textit{to} which flow extends.
The defining characteristics of these flows are that they have only radial, and no tangential, components.
Expressed mathematically, the velocity potential for source/sink flow is

\begin{marginfigure}
	\begin{tikzpicture}[/tikz/background rectangle/.style={fill={rgb,1:red,1.0;green,1.0;blue,1.0}, draw opacity={1.0}}, show background rectangle]
\begin{axis}[point meta max={nan}, point meta min={nan}, title={}, title style={at={{(0.5,1)}}, anchor={south}, font={{\fontsize{14 pt}{18.2 pt}\selectfont}}, color={rgb,1:red,0.0;green,0.0;blue,0.0}, draw opacity={1.0}, rotate={0.0}}, legend style={color={rgb,1:red,0.0;green,0.0;blue,0.0}, draw opacity={0.0}, line width={1}, solid, fill={rgb,1:red,0.0;green,0.0;blue,0.0}, fill opacity={0.0}, text opacity={1.0}, font={{\fontsize{8 pt}{10.4 pt}\selectfont}}, text={rgb,1:red,0.0;green,0.0;blue,0.0}, cells={anchor={center}}, at={(1.02, 1)}, anchor={north west}}, axis background/.style={fill={rgb,1:red,0.0;green,0.0;blue,0.0}, opacity={0.0}}, anchor={north west}, xshift={1.0mm}, yshift={-1.0mm}, width={48.8mm}, height={48.8mm}, scaled x ticks={false}, xlabel={}, x tick style={color={rgb,1:red,0.0;green,0.0;blue,0.0}, opacity={1.0}}, x tick label style={color={rgb,1:red,0.0;green,0.0;blue,0.0}, opacity={1.0}, rotate={0}}, xlabel style={at={(ticklabel cs:0.5)}, anchor=near ticklabel, font={{\fontsize{11 pt}{14.3 pt}\selectfont}}, color={rgb,1:red,0.0;green,0.0;blue,0.0}, draw opacity={1.0}, rotate={0.0}}, xmajorticks={false}, xmajorgrids={false}, xmin={0}, xmax={3}, axis x line*={left}, separate axis lines, x axis line style={{draw opacity = 0}}, scaled y ticks={false}, ylabel={}, y tick style={color={rgb,1:red,0.0;green,0.0;blue,0.0}, opacity={1.0}}, y tick label style={color={rgb,1:red,0.0;green,0.0;blue,0.0}, opacity={1.0}, rotate={0}}, ylabel style={at={(ticklabel cs:0.5)}, anchor=near ticklabel, font={{\fontsize{11 pt}{14.3 pt}\selectfont}}, color={rgb,1:red,0.0;green,0.0;blue,0.0}, draw opacity={1.0}, rotate={0.0}}, ymajorticks={false}, ymajorgrids={false}, ymin={0}, ymax={3}, axis y line*={left}, y axis line style={{draw opacity = 0}}, colorbar={false}]
    \addplot[color={rgb,1:red,0.0;green,0.1804;blue,0.3647}, name path={3a41e08d-3a87-4507-ac74-68adcccf1878}, draw opacity={1.0}, line width={2.0}, solid, mark={*}, mark size={3.0 pt}, mark repeat={1}, mark options={color={rgb,1:red,0.0;green,0.0;blue,0.0}, draw opacity={0.0}, fill={rgb,1:red,0.0;green,0.1804;blue,0.3647}, fill opacity={1.0}, line width={0.75}, rotate={0}, solid}]
        table[row sep={\\}]
        {
            \\
            1.5  1.5  \\
        }
        ;
    \addplot[color={rgb,1:red,0.0;green,0.1804;blue,0.3647}, name path={38a61fcb-6078-4a64-af08-06edda8092dc}, draw opacity={1.0}, line width={2.0}, solid, quiver={u={\thisrow{u}}, v={\thisrow{v}}, every arrow/.append style={-{stealth[length = 0.3pt, width = 0.3pt]}}}]
        table[row sep={\\}]
        {
            x  y  u  v  \\
            1.5  3.0  0.0  -1.15  \\
        }
        ;
    \addplot[color={rgb,1:red,0.0;green,0.1804;blue,0.3647}, name path={38a61fcb-6078-4a64-af08-06edda8092dc}, draw opacity={1.0}, line width={2.0}, solid]
        table[row sep={\\}]
        {
            \\
            1.5  3.0  \\
        }
        ;
    \addplot[color={rgb,1:red,0.0;green,0.1804;blue,0.3647}, name path={dd839118-715d-45dc-bb63-bdcb93bc175a}, draw opacity={1.0}, line width={2.0}, solid, quiver={u={\thisrow{u}}, v={\thisrow{v}}, every arrow/.append style={-{stealth[length = 0.3pt, width = 0.3pt]}}}]
        table[row sep={\\}]
        {
            x  y  u  v  \\
            0.6890387738166035  2.761880299246772  0.6217369400739374  -0.9674415627558586  \\
        }
        ;
    \addplot[color={rgb,1:red,0.0;green,0.1804;blue,0.3647}, name path={dd839118-715d-45dc-bb63-bdcb93bc175a}, draw opacity={1.0}, line width={2.0}, solid]
        table[row sep={\\}]
        {
            \\
            0.6890387738166035  2.761880299246772  \\
        }
        ;
    \addplot[color={rgb,1:red,0.0;green,0.1804;blue,0.3647}, name path={a52b0485-c468-479d-a84a-68a4f434a580}, draw opacity={1.0}, line width={2.0}, solid, quiver={u={\thisrow{u}}, v={\thisrow{v}}, every arrow/.append style={-{stealth[length = 0.3pt, width = 0.3pt]}}}]
        table[row sep={\\}]
        {
            x  y  u  v  \\
            0.13555200696822256  2.12312251950283  1.046076794657696  -0.4777272649521698  \\
        }
        ;
    \addplot[color={rgb,1:red,0.0;green,0.1804;blue,0.3647}, name path={a52b0485-c468-479d-a84a-68a4f434a580}, draw opacity={1.0}, line width={2.0}, solid]
        table[row sep={\\}]
        {
            \\
            0.13555200696822256  2.12312251950283  \\
        }
        ;
    \addplot[color={rgb,1:red,0.0;green,0.1804;blue,0.3647}, name path={27c4545e-7067-4cb9-988c-15a2fc2c06a0}, draw opacity={1.0}, line width={2.0}, solid, quiver={u={\thisrow{u}}, v={\thisrow{v}}, every arrow/.append style={-{stealth[length = 0.3pt, width = 0.3pt]}}}]
        table[row sep={\\}]
        {
            x  y  u  v  \\
            0.015267837178600807  1.2865277425900725  1.1382946581630726  0.16366206401427785  \\
        }
        ;
    \addplot[color={rgb,1:red,0.0;green,0.1804;blue,0.3647}, name path={27c4545e-7067-4cb9-988c-15a2fc2c06a0}, draw opacity={1.0}, line width={2.0}, solid]
        table[row sep={\\}]
        {
            \\
            0.015267837178600807  1.2865277425900725  \\
        }
        ;
    \addplot[color={rgb,1:red,0.0;green,0.1804;blue,0.3647}, name path={f3b4b8d9-7eeb-4ef1-88c9-f3184d0a19de}, draw opacity={1.0}, line width={2.0}, solid, quiver={u={\thisrow{u}}, v={\thisrow{v}}, every arrow/.append style={-{stealth[length = 0.3pt, width = 0.3pt]}}}]
        table[row sep={\\}]
        {
            x  y  u  v  \\
            0.3663756384686123  0.5177088990820725  0.8691120105073973  0.7530898440370778  \\
        }
        ;
    \addplot[color={rgb,1:red,0.0;green,0.1804;blue,0.3647}, name path={f3b4b8d9-7eeb-4ef1-88c9-f3184d0a19de}, draw opacity={1.0}, line width={2.0}, solid]
        table[row sep={\\}]
        {
            \\
            0.3663756384686123  0.5177088990820725  \\
        }
        ;
    \addplot[color={rgb,1:red,0.0;green,0.1804;blue,0.3647}, name path={0df22b81-e882-4e3e-a7d6-cf198782159f}, draw opacity={1.0}, line width={2.0}, solid, quiver={u={\thisrow{u}}, v={\thisrow{v}}, every arrow/.append style={-{stealth[length = 0.3pt, width = 0.3pt]}}}]
        table[row sep={\\}]
        {
            x  y  u  v  \\
            1.0774011647378554  0.060760539578254  0.3239924403676442  1.1034169196566719  \\
        }
        ;
    \addplot[color={rgb,1:red,0.0;green,0.1804;blue,0.3647}, name path={0df22b81-e882-4e3e-a7d6-cf198782159f}, draw opacity={1.0}, line width={2.0}, solid]
        table[row sep={\\}]
        {
            \\
            1.0774011647378554  0.060760539578254  \\
        }
        ;
    \addplot[color={rgb,1:red,0.0;green,0.1804;blue,0.3647}, name path={a79f4aeb-028a-4061-b0d3-7d459614280c}, draw opacity={1.0}, line width={2.0}, solid, quiver={u={\thisrow{u}}, v={\thisrow{v}}, every arrow/.append style={-{stealth[length = 0.3pt, width = 0.3pt]}}}]
        table[row sep={\\}]
        {
            x  y  u  v  \\
            1.9225988352621441  0.06076053957825378  -0.32399244036764396  1.103416919656672  \\
        }
        ;
    \addplot[color={rgb,1:red,0.0;green,0.1804;blue,0.3647}, name path={a79f4aeb-028a-4061-b0d3-7d459614280c}, draw opacity={1.0}, line width={2.0}, solid]
        table[row sep={\\}]
        {
            \\
            1.9225988352621441  0.06076053957825378  \\
        }
        ;
    \addplot[color={rgb,1:red,0.0;green,0.1804;blue,0.3647}, name path={84e63717-435c-4cc8-94e7-f92aec096d27}, draw opacity={1.0}, line width={2.0}, solid, quiver={u={\thisrow{u}}, v={\thisrow{v}}, every arrow/.append style={-{stealth[length = 0.3pt, width = 0.3pt]}}}]
        table[row sep={\\}]
        {
            x  y  u  v  \\
            2.6336243615313877  0.5177088990820726  -0.8691120105073973  0.7530898440370777  \\
        }
        ;
    \addplot[color={rgb,1:red,0.0;green,0.1804;blue,0.3647}, name path={84e63717-435c-4cc8-94e7-f92aec096d27}, draw opacity={1.0}, line width={2.0}, solid]
        table[row sep={\\}]
        {
            \\
            2.6336243615313877  0.5177088990820726  \\
        }
        ;
    \addplot[color={rgb,1:red,0.0;green,0.1804;blue,0.3647}, name path={6f7a49a3-fc7c-4425-9b84-b5916cbf8f71}, draw opacity={1.0}, line width={2.0}, solid, quiver={u={\thisrow{u}}, v={\thisrow{v}}, every arrow/.append style={-{stealth[length = 0.3pt, width = 0.3pt]}}}]
        table[row sep={\\}]
        {
            x  y  u  v  \\
            2.984732162821399  1.286527742590072  -1.1382946581630728  0.16366206401427807  \\
        }
        ;
    \addplot[color={rgb,1:red,0.0;green,0.1804;blue,0.3647}, name path={6f7a49a3-fc7c-4425-9b84-b5916cbf8f71}, draw opacity={1.0}, line width={2.0}, solid]
        table[row sep={\\}]
        {
            \\
            2.984732162821399  1.286527742590072  \\
        }
        ;
    \addplot[color={rgb,1:red,0.0;green,0.1804;blue,0.3647}, name path={869eb2a6-a725-45bf-84b0-a096065de24a}, draw opacity={1.0}, line width={2.0}, solid, quiver={u={\thisrow{u}}, v={\thisrow{v}}, every arrow/.append style={-{stealth[length = 0.3pt, width = 0.3pt]}}}]
        table[row sep={\\}]
        {
            x  y  u  v  \\
            2.864447993031778  2.1231225195028287  -1.0460767946576963  -0.47772726495216866  \\
        }
        ;
    \addplot[color={rgb,1:red,0.0;green,0.1804;blue,0.3647}, name path={869eb2a6-a725-45bf-84b0-a096065de24a}, draw opacity={1.0}, line width={2.0}, solid]
        table[row sep={\\}]
        {
            \\
            2.864447993031778  2.1231225195028287  \\
        }
        ;
    \addplot[color={rgb,1:red,0.0;green,0.1804;blue,0.3647}, name path={44509def-3198-4f66-ac6c-38914514cf98}, draw opacity={1.0}, line width={2.0}, solid, quiver={u={\thisrow{u}}, v={\thisrow{v}}, every arrow/.append style={-{stealth[length = 0.3pt, width = 0.3pt]}}}]
        table[row sep={\\}]
        {
            x  y  u  v  \\
            2.3109612261833963  2.761880299246772  -0.6217369400739372  -0.9674415627558586  \\
        }
        ;
    \addplot[color={rgb,1:red,0.0;green,0.1804;blue,0.3647}, name path={44509def-3198-4f66-ac6c-38914514cf98}, draw opacity={1.0}, line width={2.0}, solid]
        table[row sep={\\}]
        {
            \\
            2.3109612261833963  2.761880299246772  \\
        }
        ;
    \addplot[color={rgb,1:red,0.0;green,0.1804;blue,0.3647}, name path={5e5776ab-0b24-483a-9cb6-a11b85ad133c}, draw opacity={1.0}, line width={2.0}, solid, quiver={u={\thisrow{u}}, v={\thisrow{v}}, every arrow/.append style={-{stealth[length = 0.3pt, width = 0.3pt]}}}]
        table[row sep={\\}]
        {
            x  y  u  v  \\
            1.5000000000000004  3.0  -4.440892098500626e-16  -1.15  \\
        }
        ;
    \addplot[color={rgb,1:red,0.0;green,0.1804;blue,0.3647}, name path={5e5776ab-0b24-483a-9cb6-a11b85ad133c}, draw opacity={1.0}, line width={2.0}, solid]
        table[row sep={\\}]
        {
            \\
            1.5000000000000004  3.0  \\
        }
        ;
\end{axis}
\end{tikzpicture}

\end{marginfigure}


\begin{equation}
\label{eqn:sourceflow}
	\phi_s = \frac{\pm\Lambda}{2\pi} \ln(r)
\end{equation}

\noindent where \(\Lambda\) is the strength of the source/sink and \(r\) is the radial distance from the source/sink.
A positive sign indicates a source, and a negative, a sink.


\subsubsection{Doublet Flow}

\begin{marginfigure}
	\begin{tikzpicture}[/tikz/background rectangle/.style={fill={rgb,1:red,1.0;green,1.0;blue,1.0}, draw opacity={1.0}}, show background rectangle]
\begin{axis}[point meta max={nan}, point meta min={nan}, title={}, title style={at={{(0.5,1)}}, anchor={south}, font={{\fontsize{14 pt}{18.2 pt}\selectfont}}, color={rgb,1:red,0.0;green,0.0;blue,0.0}, draw opacity={1.0}, rotate={0.0}}, legend style={color={rgb,1:red,0.0;green,0.0;blue,0.0}, draw opacity={0.0}, line width={1}, solid, fill={rgb,1:red,0.0;green,0.0;blue,0.0}, fill opacity={0.0}, text opacity={1.0}, font={{\fontsize{8 pt}{10.4 pt}\selectfont}}, text={rgb,1:red,0.0;green,0.0;blue,0.0}, cells={anchor={center}}, at={(1.02, 1)}, anchor={north west}}, axis background/.style={fill={rgb,1:red,0.0;green,0.0;blue,0.0}, opacity={0.0}}, anchor={north west}, xshift={1.0mm}, yshift={-1.0mm}, width={48.8mm}, height={48.8mm}, scaled x ticks={false}, xlabel={}, x tick style={color={rgb,1:red,0.0;green,0.0;blue,0.0}, opacity={1.0}}, x tick label style={color={rgb,1:red,0.0;green,0.0;blue,0.0}, opacity={1.0}, rotate={0}}, xlabel style={at={(ticklabel cs:0.5)}, anchor=near ticklabel, font={{\fontsize{11 pt}{14.3 pt}\selectfont}}, color={rgb,1:red,0.0;green,0.0;blue,0.0}, draw opacity={1.0}, rotate={0.0}}, xmajorticks={false}, xmajorgrids={false}, xmin={-1.5}, xmax={1.5}, axis x line*={left}, separate axis lines, x axis line style={{draw opacity = 0}}, scaled y ticks={false}, ylabel={}, y tick style={color={rgb,1:red,0.0;green,0.0;blue,0.0}, opacity={1.0}}, y tick label style={color={rgb,1:red,0.0;green,0.0;blue,0.0}, opacity={1.0}, rotate={0}}, ylabel style={at={(ticklabel cs:0.5)}, anchor=near ticklabel, font={{\fontsize{11 pt}{14.3 pt}\selectfont}}, color={rgb,1:red,0.0;green,0.0;blue,0.0}, draw opacity={1.0}, rotate={0.0}}, ymajorticks={false}, ymajorgrids={false}, ymin={-1.5}, ymax={1.5}, axis y line*={left}, y axis line style={{draw opacity = 0}}, colorbar={false}]
    \addplot[color={rgb,1:red,0.0;green,0.1804;blue,0.3647}, name path={ec02a32b-3509-49ba-a2e4-468a728331d5}, draw opacity={1.0}, line width={2.0}, solid, mark={*}, mark size={3.0 pt}, mark repeat={1}, mark options={color={rgb,1:red,0.0;green,0.0;blue,0.0}, draw opacity={0.0}, fill={rgb,1:red,0.0;green,0.1804;blue,0.3647}, fill opacity={1.0}, line width={0.75}, rotate={0}, solid}]
        table[row sep={\\}]
        {
            \\
            0.0  0.0  \\
        }
        ;
    \addplot[color={rgb,1:red,0.0;green,0.1804;blue,0.3647}, name path={165f9e19-5bbe-4ef5-9334-ec7ed9fb5b2c}, draw opacity={1.0}, line width={2.0}, solid, quiver={u={\thisrow{u}}, v={\thisrow{v}}, every arrow/.append style={-{stealth[length = 0.3pt, width = 0.3pt]}}}]
        table[row sep={\\}]
        {
            x  y  u  v  \\
            -0.052706223612201446  0.29043523059596055  0.0259226893924059  0.007154207693833903  \\
        }
        ;
    \addplot[color={rgb,1:red,0.0;green,0.1804;blue,0.3647}, name path={165f9e19-5bbe-4ef5-9334-ec7ed9fb5b2c}, draw opacity={1.0}, line width={2.0}, solid]
        table[row sep={\\}]
        {
            \\
            -9.184850993605149e-18  0.3  \\
            0.02678353421979549  0.29758943828979445  \\
            0.05270622361220139  0.29043523059596055  \\
            0.0769348916108859  0.2787673190402799  \\
            0.09869080889095688  0.26296071990054165  \\
            0.11727472237020445  0.24352347027881005  \\
            0.13208932977851068  0.2210802993709498  \\
            0.142658477444273  0.19635254915624212  \\
            0.1486424642651902  0.17013498987264833  \\
            0.1498489599811972  0.14327027544742277  \\
            0.14623918682727355  0.11662185990655285  \\
            0.13792916588271759  0.09104624525191148  \\
            0.12518598805819542  0.06736545278218463  \\
            0.10841922957410875  0.04634060265197032  \\
            0.08816778784387098  0.028647450843757902  \\
            0.06508256086763374  0.01485466981463715  \\
            0.039905526835001294  0.005405570895622019  \\
            0.01344589633551505  0.0006038559007141286  \\
            -0.013445896335514997  0.0006038559007141286  \\
            -0.03990552683500123  0.005405570895621992  \\
            -0.06508256086763368  0.014854669814637123  \\
            -0.08816778784387093  0.02864745084375786  \\
            -0.1084192295741087  0.04634060265197028  \\
            -0.12518598805819536  0.06736545278218459  \\
            -0.13792916588271759  0.09104624525191142  \\
            -0.14623918682727352  0.1166218599065528  \\
            -0.14984895998119718  0.1432702754474227  \\
            -0.1486424642651902  0.17013498987264827  \\
            -0.14265847744427304  0.19635254915624206  \\
            -0.1320893297785107  0.22108029937094975  \\
            -0.11727472237020448  0.24352347027881  \\
            -0.09869080889095692  0.2629607199005416  \\
            -0.07693489161088594  0.27876731904027985  \\
            -0.052706223612201446  0.29043523059596055  \\
        }
        ;
    \addplot[color={rgb,1:red,0.0;green,0.1804;blue,0.3647}, name path={e3a88e7e-d062-4642-b49f-928b4075c5cb}, draw opacity={1.0}, line width={2.0}, solid, quiver={u={\thisrow{u}}, v={\thisrow{v}}, every arrow/.append style={-{stealth[length = 0.3pt, width = 0.3pt]}}}]
        table[row sep={\\}]
        {
            x  y  u  v  \\
            -0.052706223612201446  -0.29043523059596055  0.0259226893924059  -0.007154207693833903  \\
        }
        ;
    \addplot[color={rgb,1:red,0.0;green,0.1804;blue,0.3647}, name path={e3a88e7e-d062-4642-b49f-928b4075c5cb}, draw opacity={1.0}, line width={2.0}, solid]
        table[row sep={\\}]
        {
            \\
            -9.184850993605149e-18  -0.3  \\
            0.02678353421979549  -0.29758943828979445  \\
            0.05270622361220139  -0.29043523059596055  \\
            0.0769348916108859  -0.2787673190402799  \\
            0.09869080889095688  -0.26296071990054165  \\
            0.11727472237020445  -0.24352347027881005  \\
            0.13208932977851068  -0.2210802993709498  \\
            0.142658477444273  -0.19635254915624212  \\
            0.1486424642651902  -0.17013498987264833  \\
            0.1498489599811972  -0.14327027544742277  \\
            0.14623918682727355  -0.11662185990655285  \\
            0.13792916588271759  -0.09104624525191148  \\
            0.12518598805819542  -0.06736545278218463  \\
            0.10841922957410875  -0.04634060265197032  \\
            0.08816778784387098  -0.028647450843757902  \\
            0.06508256086763374  -0.01485466981463715  \\
            0.039905526835001294  -0.005405570895622019  \\
            0.01344589633551505  -0.0006038559007141286  \\
            -0.013445896335514997  -0.0006038559007141286  \\
            -0.03990552683500123  -0.005405570895621992  \\
            -0.06508256086763368  -0.014854669814637123  \\
            -0.08816778784387093  -0.02864745084375786  \\
            -0.1084192295741087  -0.04634060265197028  \\
            -0.12518598805819536  -0.06736545278218459  \\
            -0.13792916588271759  -0.09104624525191142  \\
            -0.14623918682727352  -0.1166218599065528  \\
            -0.14984895998119718  -0.1432702754474227  \\
            -0.1486424642651902  -0.17013498987264827  \\
            -0.14265847744427304  -0.19635254915624206  \\
            -0.1320893297785107  -0.22108029937094975  \\
            -0.11727472237020448  -0.24352347027881  \\
            -0.09869080889095692  -0.2629607199005416  \\
            -0.07693489161088594  -0.27876731904027985  \\
            -0.052706223612201446  -0.29043523059596055  \\
        }
        ;
    \addplot[color={rgb,1:red,0.0;green,0.1804;blue,0.3647}, name path={2860666a-f696-49d1-8f6d-eeec0aa9ebd7}, draw opacity={1.0}, line width={2.0}, solid, quiver={u={\thisrow{u}}, v={\thisrow{v}}, every arrow/.append style={-{stealth[length = 0.3pt, width = 0.3pt]}}}]
        table[row sep={\\}]
        {
            x  y  u  v  \\
            -0.10541244722440289  0.5808704611919211  0.0518453787848118  0.014308415387667806  \\
        }
        ;
    \addplot[color={rgb,1:red,0.0;green,0.1804;blue,0.3647}, name path={2860666a-f696-49d1-8f6d-eeec0aa9ebd7}, draw opacity={1.0}, line width={2.0}, solid]
        table[row sep={\\}]
        {
            \\
            -1.8369701987210297e-17  0.6  \\
            0.05356706843959098  0.5951788765795889  \\
            0.10541244722440278  0.5808704611919211  \\
            0.1538697832217718  0.5575346380805598  \\
            0.19738161778191377  0.5259214398010833  \\
            0.2345494447404089  0.4870469405576201  \\
            0.26417865955702136  0.4421605987418996  \\
            0.285316954888546  0.39270509831248424  \\
            0.2972849285303804  0.34026997974529666  \\
            0.2996979199623944  0.28654055089484554  \\
            0.2924783736545471  0.2332437198131057  \\
            0.27585833176543517  0.18209249050382295  \\
            0.25037197611639084  0.13473090556436926  \\
            0.2168384591482175  0.09268120530394064  \\
            0.17633557568774197  0.057294901687515803  \\
            0.13016512173526748  0.0297093396292743  \\
            0.07981105367000259  0.010811141791244039  \\
            0.0268917926710301  0.0012077118014282573  \\
            -0.026891792671029993  0.0012077118014282573  \\
            -0.07981105367000246  0.010811141791243983  \\
            -0.13016512173526737  0.029709339629274245  \\
            -0.17633557568774186  0.05729490168751572  \\
            -0.2168384591482174  0.09268120530394056  \\
            -0.2503719761163907  0.13473090556436917  \\
            -0.27585833176543517  0.18209249050382284  \\
            -0.29247837365454704  0.2332437198131056  \\
            -0.29969791996239437  0.2865405508948454  \\
            -0.2972849285303804  0.34026997974529655  \\
            -0.2853169548885461  0.3927050983124841  \\
            -0.2641786595570214  0.4421605987418995  \\
            -0.23454944474040895  0.48704694055762  \\
            -0.19738161778191385  0.5259214398010832  \\
            -0.1538697832217719  0.5575346380805597  \\
            -0.10541244722440289  0.5808704611919211  \\
        }
        ;
    \addplot[color={rgb,1:red,0.0;green,0.1804;blue,0.3647}, name path={34ad3012-57d8-454b-b87a-78b3e93eff63}, draw opacity={1.0}, line width={2.0}, solid, quiver={u={\thisrow{u}}, v={\thisrow{v}}, every arrow/.append style={-{stealth[length = 0.3pt, width = 0.3pt]}}}]
        table[row sep={\\}]
        {
            x  y  u  v  \\
            -0.10541244722440289  -0.5808704611919211  0.0518453787848118  -0.014308415387667806  \\
        }
        ;
    \addplot[color={rgb,1:red,0.0;green,0.1804;blue,0.3647}, name path={34ad3012-57d8-454b-b87a-78b3e93eff63}, draw opacity={1.0}, line width={2.0}, solid]
        table[row sep={\\}]
        {
            \\
            -1.8369701987210297e-17  -0.6  \\
            0.05356706843959098  -0.5951788765795889  \\
            0.10541244722440278  -0.5808704611919211  \\
            0.1538697832217718  -0.5575346380805598  \\
            0.19738161778191377  -0.5259214398010833  \\
            0.2345494447404089  -0.4870469405576201  \\
            0.26417865955702136  -0.4421605987418996  \\
            0.285316954888546  -0.39270509831248424  \\
            0.2972849285303804  -0.34026997974529666  \\
            0.2996979199623944  -0.28654055089484554  \\
            0.2924783736545471  -0.2332437198131057  \\
            0.27585833176543517  -0.18209249050382295  \\
            0.25037197611639084  -0.13473090556436926  \\
            0.2168384591482175  -0.09268120530394064  \\
            0.17633557568774197  -0.057294901687515803  \\
            0.13016512173526748  -0.0297093396292743  \\
            0.07981105367000259  -0.010811141791244039  \\
            0.0268917926710301  -0.0012077118014282573  \\
            -0.026891792671029993  -0.0012077118014282573  \\
            -0.07981105367000246  -0.010811141791243983  \\
            -0.13016512173526737  -0.029709339629274245  \\
            -0.17633557568774186  -0.05729490168751572  \\
            -0.2168384591482174  -0.09268120530394056  \\
            -0.2503719761163907  -0.13473090556436917  \\
            -0.27585833176543517  -0.18209249050382284  \\
            -0.29247837365454704  -0.2332437198131056  \\
            -0.29969791996239437  -0.2865405508948454  \\
            -0.2972849285303804  -0.34026997974529655  \\
            -0.2853169548885461  -0.3927050983124841  \\
            -0.2641786595570214  -0.4421605987418995  \\
            -0.23454944474040895  -0.48704694055762  \\
            -0.19738161778191385  -0.5259214398010832  \\
            -0.1538697832217719  -0.5575346380805597  \\
            -0.10541244722440289  -0.5808704611919211  \\
        }
        ;
    \addplot[color={rgb,1:red,0.0;green,0.1804;blue,0.3647}, name path={d9c5a6d3-a023-48e2-95ac-a9685c53c746}, draw opacity={1.0}, line width={2.0}, solid, quiver={u={\thisrow{u}}, v={\thisrow{v}}, every arrow/.append style={-{stealth[length = 0.3pt, width = 0.3pt]}}}]
        table[row sep={\\}]
        {
            x  y  u  v  \\
            -0.15811867083660433  0.8713056917878816  0.07776806817721771  0.021462623081501597  \\
        }
        ;
    \addplot[color={rgb,1:red,0.0;green,0.1804;blue,0.3647}, name path={d9c5a6d3-a023-48e2-95ac-a9685c53c746}, draw opacity={1.0}, line width={2.0}, solid]
        table[row sep={\\}]
        {
            \\
            -2.7554552980815445e-17  0.8999999999999999  \\
            0.08035060265938646  0.8927683148693832  \\
            0.15811867083660416  0.8713056917878816  \\
            0.23080467483265768  0.8363019571208397  \\
            0.29607242667287065  0.7888821597016249  \\
            0.3518241671106133  0.73057041083643  \\
            0.39626798933553203  0.6632408981128494  \\
            0.42797543233281904  0.5890576474687264  \\
            0.4459273927955706  0.510404969617945  \\
            0.44954687994359155  0.4298108263422683  \\
            0.4387175604818206  0.34986557971965854  \\
            0.41378749764815276  0.27313873575573444  \\
            0.3755579641745862  0.2020963583465539  \\
            0.32525768872232624  0.13902180795591096  \\
            0.2645033635316129  0.08594235253127369  \\
            0.1952476826029012  0.04456400944391142  \\
            0.11971658050500386  0.01621671268686603  \\
            0.04033768900654515  0.0018115677021424137  \\
            -0.04033768900654499  0.0018115677021424137  \\
            -0.1197165805050037  0.016216712686865975  \\
            -0.19524768260290107  0.04456400944391131  \\
            -0.2645033635316128  0.08594235253127358  \\
            -0.3252576887223261  0.13902180795591085  \\
            -0.3755579641745861  0.20209635834655373  \\
            -0.4137874976481527  0.2731387357557342  \\
            -0.43871756048182053  0.34986557971965837  \\
            -0.44954687994359155  0.42981082634226814  \\
            -0.44592739279557064  0.5104049696179448  \\
            -0.4279754323328191  0.5890576474687261  \\
            -0.3962679893355321  0.6632408981128493  \\
            -0.35182416711061343  0.7305704108364299  \\
            -0.29607242667287076  0.7888821597016247  \\
            -0.23080467483265785  0.8363019571208397  \\
            -0.15811867083660433  0.8713056917878816  \\
        }
        ;
    \addplot[color={rgb,1:red,0.0;green,0.1804;blue,0.3647}, name path={e9e5f08e-0452-49fe-a480-fb776329f780}, draw opacity={1.0}, line width={2.0}, solid, quiver={u={\thisrow{u}}, v={\thisrow{v}}, every arrow/.append style={-{stealth[length = 0.3pt, width = 0.3pt]}}}]
        table[row sep={\\}]
        {
            x  y  u  v  \\
            -0.15811867083660433  -0.8713056917878816  0.07776806817721771  -0.021462623081501597  \\
        }
        ;
    \addplot[color={rgb,1:red,0.0;green,0.1804;blue,0.3647}, name path={e9e5f08e-0452-49fe-a480-fb776329f780}, draw opacity={1.0}, line width={2.0}, solid]
        table[row sep={\\}]
        {
            \\
            -2.7554552980815445e-17  -0.8999999999999999  \\
            0.08035060265938646  -0.8927683148693832  \\
            0.15811867083660416  -0.8713056917878816  \\
            0.23080467483265768  -0.8363019571208397  \\
            0.29607242667287065  -0.7888821597016249  \\
            0.3518241671106133  -0.73057041083643  \\
            0.39626798933553203  -0.6632408981128494  \\
            0.42797543233281904  -0.5890576474687264  \\
            0.4459273927955706  -0.510404969617945  \\
            0.44954687994359155  -0.4298108263422683  \\
            0.4387175604818206  -0.34986557971965854  \\
            0.41378749764815276  -0.27313873575573444  \\
            0.3755579641745862  -0.2020963583465539  \\
            0.32525768872232624  -0.13902180795591096  \\
            0.2645033635316129  -0.08594235253127369  \\
            0.1952476826029012  -0.04456400944391142  \\
            0.11971658050500386  -0.01621671268686603  \\
            0.04033768900654515  -0.0018115677021424137  \\
            -0.04033768900654499  -0.0018115677021424137  \\
            -0.1197165805050037  -0.016216712686865975  \\
            -0.19524768260290107  -0.04456400944391131  \\
            -0.2645033635316128  -0.08594235253127358  \\
            -0.3252576887223261  -0.13902180795591085  \\
            -0.3755579641745861  -0.20209635834655373  \\
            -0.4137874976481527  -0.2731387357557342  \\
            -0.43871756048182053  -0.34986557971965837  \\
            -0.44954687994359155  -0.42981082634226814  \\
            -0.44592739279557064  -0.5104049696179448  \\
            -0.4279754323328191  -0.5890576474687261  \\
            -0.3962679893355321  -0.6632408981128493  \\
            -0.35182416711061343  -0.7305704108364299  \\
            -0.29607242667287076  -0.7888821597016247  \\
            -0.23080467483265785  -0.8363019571208397  \\
            -0.15811867083660433  -0.8713056917878816  \\
        }
        ;
    \addplot[color={rgb,1:red,0.0;green,0.1804;blue,0.3647}, name path={d289c109-0561-43ae-8549-1a3f1854f034}, draw opacity={1.0}, line width={2.0}, solid, quiver={u={\thisrow{u}}, v={\thisrow{v}}, every arrow/.append style={-{stealth[length = 0.3pt, width = 0.3pt]}}}]
        table[row sep={\\}]
        {
            x  y  u  v  \\
            -0.21082489444880578  1.1617409223838422  0.1036907575696236  0.02861683077533561  \\
        }
        ;
    \addplot[color={rgb,1:red,0.0;green,0.1804;blue,0.3647}, name path={d289c109-0561-43ae-8549-1a3f1854f034}, draw opacity={1.0}, line width={2.0}, solid]
        table[row sep={\\}]
        {
            \\
            -3.6739403974420595e-17  1.2  \\
            0.10713413687918195  1.1903577531591778  \\
            0.21082489444880556  1.1617409223838422  \\
            0.3077395664435436  1.1150692761611196  \\
            0.39476323556382753  1.0518428796021666  \\
            0.4690988894808178  0.9740938811152402  \\
            0.5283573191140427  0.8843211974837992  \\
            0.570633909777092  0.7854101966249685  \\
            0.5945698570607608  0.6805399594905933  \\
            0.5993958399247888  0.5730811017896911  \\
            0.5849567473090942  0.4664874396262114  \\
            0.5517166635308703  0.3641849810076459  \\
            0.5007439522327817  0.2694618111287385  \\
            0.433676918296435  0.18536241060788128  \\
            0.35267115137548394  0.11458980337503161  \\
            0.26033024347053496  0.0594186792585486  \\
            0.15962210734000518  0.021622283582488078  \\
            0.0537835853420602  0.0024154236028565146  \\
            -0.053783585342059986  0.0024154236028565146  \\
            -0.15962210734000493  0.021622283582487967  \\
            -0.26033024347053474  0.05941867925854849  \\
            -0.3526711513754837  0.11458980337503144  \\
            -0.4336769182964348  0.1853624106078811  \\
            -0.5007439522327815  0.26946181112873835  \\
            -0.5517166635308703  0.3641849810076457  \\
            -0.5849567473090941  0.4664874396262112  \\
            -0.5993958399247887  0.5730811017896908  \\
            -0.5945698570607608  0.6805399594905931  \\
            -0.5706339097770922  0.7854101966249682  \\
            -0.5283573191140428  0.884321197483799  \\
            -0.4690988894808179  0.97409388111524  \\
            -0.3947632355638277  1.0518428796021664  \\
            -0.3077395664435438  1.1150692761611194  \\
            -0.21082489444880578  1.1617409223838422  \\
        }
        ;
    \addplot[color={rgb,1:red,0.0;green,0.1804;blue,0.3647}, name path={3539b91f-e0f3-4534-b90e-26f97882a529}, draw opacity={1.0}, line width={2.0}, solid, quiver={u={\thisrow{u}}, v={\thisrow{v}}, every arrow/.append style={-{stealth[length = 0.3pt, width = 0.3pt]}}}]
        table[row sep={\\}]
        {
            x  y  u  v  \\
            -0.21082489444880578  -1.1617409223838422  0.1036907575696236  -0.02861683077533561  \\
        }
        ;
    \addplot[color={rgb,1:red,0.0;green,0.1804;blue,0.3647}, name path={3539b91f-e0f3-4534-b90e-26f97882a529}, draw opacity={1.0}, line width={2.0}, solid]
        table[row sep={\\}]
        {
            \\
            -3.6739403974420595e-17  -1.2  \\
            0.10713413687918195  -1.1903577531591778  \\
            0.21082489444880556  -1.1617409223838422  \\
            0.3077395664435436  -1.1150692761611196  \\
            0.39476323556382753  -1.0518428796021666  \\
            0.4690988894808178  -0.9740938811152402  \\
            0.5283573191140427  -0.8843211974837992  \\
            0.570633909777092  -0.7854101966249685  \\
            0.5945698570607608  -0.6805399594905933  \\
            0.5993958399247888  -0.5730811017896911  \\
            0.5849567473090942  -0.4664874396262114  \\
            0.5517166635308703  -0.3641849810076459  \\
            0.5007439522327817  -0.2694618111287385  \\
            0.433676918296435  -0.18536241060788128  \\
            0.35267115137548394  -0.11458980337503161  \\
            0.26033024347053496  -0.0594186792585486  \\
            0.15962210734000518  -0.021622283582488078  \\
            0.0537835853420602  -0.0024154236028565146  \\
            -0.053783585342059986  -0.0024154236028565146  \\
            -0.15962210734000493  -0.021622283582487967  \\
            -0.26033024347053474  -0.05941867925854849  \\
            -0.3526711513754837  -0.11458980337503144  \\
            -0.4336769182964348  -0.1853624106078811  \\
            -0.5007439522327815  -0.26946181112873835  \\
            -0.5517166635308703  -0.3641849810076457  \\
            -0.5849567473090941  -0.4664874396262112  \\
            -0.5993958399247887  -0.5730811017896908  \\
            -0.5945698570607608  -0.6805399594905931  \\
            -0.5706339097770922  -0.7854101966249682  \\
            -0.5283573191140428  -0.884321197483799  \\
            -0.4690988894808179  -0.97409388111524  \\
            -0.3947632355638277  -1.0518428796021664  \\
            -0.3077395664435438  -1.1150692761611194  \\
            -0.21082489444880578  -1.1617409223838422  \\
        }
        ;
\end{axis}
\end{tikzpicture}

\end{marginfigure}

Taking a source and sink with equal strengths and moving them toward each other until infinitesimally close together does not cancel them out, but rather induces a twin-lobbed flow field.
Rather than treating them separately, this combination creates it's own elementary flow that we call a doublet.
The velocity potential for a doublet is described by

\begin{equation}
\label{eqn:doubletflow}
	\phi_d = \frac{\kappa}{2\pi} \frac{\cos\theta}{r}
\end{equation}

\noindent, where \(\kappa\) is the doublet strength and \(\theta\) is the angle relative to the doublet axis.

\subsubsection{Vortex Flow}

The final elementary flow that we will discuss here is the vortex.
Vortex flow characteristics are the inverse of source/sink flows in that there is no radial, only tangential components to the flow.
The velocity potential is therefore similar to that of a source/sink:

\begin{marginfigure}
	\begin{tikzpicture}[/tikz/background rectangle/.style={fill={rgb,1:red,1.0;green,1.0;blue,1.0}, draw opacity={1.0}}, show background rectangle]
\begin{axis}[point meta max={nan}, point meta min={nan}, title={}, title style={at={{(0.5,1)}}, anchor={south}, font={{\fontsize{14 pt}{18.2 pt}\selectfont}}, color={rgb,1:red,0.0;green,0.0;blue,0.0}, draw opacity={1.0}, rotate={0.0}}, legend style={color={rgb,1:red,0.0;green,0.0;blue,0.0}, draw opacity={0.0}, line width={1}, solid, fill={rgb,1:red,0.0;green,0.0;blue,0.0}, fill opacity={0.0}, text opacity={1.0}, font={{\fontsize{8 pt}{10.4 pt}\selectfont}}, text={rgb,1:red,0.0;green,0.0;blue,0.0}, cells={anchor={center}}, at={(1.02, 1)}, anchor={north west}}, axis background/.style={fill={rgb,1:red,0.0;green,0.0;blue,0.0}, opacity={0.0}}, anchor={north west}, xshift={1.0mm}, yshift={-1.0mm}, width={48.8mm}, height={48.8mm}, scaled x ticks={false}, xlabel={}, x tick style={color={rgb,1:red,0.0;green,0.0;blue,0.0}, opacity={1.0}}, x tick label style={color={rgb,1:red,0.0;green,0.0;blue,0.0}, opacity={1.0}, rotate={0}}, xlabel style={at={(ticklabel cs:0.5)}, anchor=near ticklabel, font={{\fontsize{11 pt}{14.3 pt}\selectfont}}, color={rgb,1:red,0.0;green,0.0;blue,0.0}, draw opacity={1.0}, rotate={0.0}}, xmajorticks={false}, xmajorgrids={false}, xmin={0}, xmax={3}, axis x line*={left}, separate axis lines, x axis line style={{draw opacity = 0}}, scaled y ticks={false}, ylabel={}, y tick style={color={rgb,1:red,0.0;green,0.0;blue,0.0}, opacity={1.0}}, y tick label style={color={rgb,1:red,0.0;green,0.0;blue,0.0}, opacity={1.0}, rotate={0}}, ylabel style={at={(ticklabel cs:0.5)}, anchor=near ticklabel, font={{\fontsize{11 pt}{14.3 pt}\selectfont}}, color={rgb,1:red,0.0;green,0.0;blue,0.0}, draw opacity={1.0}, rotate={0.0}}, ymajorticks={false}, ymajorgrids={false}, ymin={0}, ymax={3}, axis y line*={left}, y axis line style={{draw opacity = 0}}, colorbar={false}]
    \addplot[color={rgb,1:red,0.0;green,0.1804;blue,0.3647}, name path={f37c5ae2-24ed-4e5e-a9ed-83fe657baf9b}, draw opacity={1.0}, line width={2.0}, solid, mark={*}, mark size={3.0 pt}, mark repeat={1}, mark options={color={rgb,1:red,0.0;green,0.0;blue,0.0}, draw opacity={0.0}, fill={rgb,1:red,0.0;green,0.1804;blue,0.3647}, fill opacity={1.0}, line width={0.75}, rotate={0}, solid}]
        table[row sep={\\}]
        {
            \\
            1.5  1.5  \\
        }
        ;
    \addplot[color={rgb,1:red,0.0;green,0.1804;blue,0.3647}, name path={3386ce98-a8b1-4797-ba48-575f13e3f421}, draw opacity={1.0}, line width={2.0}, solid, quiver={u={\thisrow{u}}, v={\thisrow{v}}, every arrow/.append style={-{stealth[length = 0.3pt, width = 0.3pt]}}}]
        table[row sep={\\}]
        {
            x  y  u  v  \\
            1.5552083258781033  1.8456183455109656  -0.018317065912106356  0.0024319884037158346  \\
        }
        ;
    \addplot[color={rgb,1:red,0.0;green,0.1804;blue,0.3647}, name path={3386ce98-a8b1-4797-ba48-575f13e3f421}, draw opacity={1.0}, line width={2.0}, solid]
        table[row sep={\\}]
        {
            \\
            1.5  1.85  \\
            1.481528628504338  1.8495122436125366  \\
            1.4631087400340033  1.8480503339146814  \\
            1.444791674121897  1.8456183455109656  \\
            1.4266284837160088  1.8422230567892643  \\
            1.4086697928856942  1.8378739310282135  \\
            1.390965655723313  1.8325830900213553  \\
            1.3735654168344964  1.8263652802915245  \\
            1.3565175738058812  1.8192378319896458  \\
            1.339869642033637  1.8112206105924935  \\
            1.3236680222895347  1.8023359615340457  \\
            1.3079578713936741  1.7926086479247518  \\
            1.2927829763543266  1.7820657815323022  \\
            1.2781856323256893  1.7707367472162687  \\
            1.264206524723705  1.7586531210272307  \\
            1.2508846158285103  1.7458485821986596  \\
            1.2382570361895726  1.732358819276853  \\
            1.2263589811361897  1.7182214306505568  \\
            1.215223612681794  1.7034758197575108  \\
            1.2048819670954733  1.688163085260002  \\
            1.1953628683983233  1.6723259064955198  \\
            1.1866928480257282  1.6560084245217883  \\
            1.1788960708794953  1.6392561190877224  \\
            1.1719942679759372  1.6221156818732083  \\
            1.166006675877632  1.6046348863510198  \\
            1.160949983077674  1.5868624546335783  \\
            1.1568382834858497  1.5688479216756899  \\
            1.1536830371463833  1.550641497211739  \\
            1.1514930382967379  1.5322939258121557  \\
            1.1502743908564994  1.5138563454491945  \\
            1.1500304914146562  1.4953801449662345  \\
            1.1507620197626982  1.4769168208478567  \\
            1.1524669369999143  1.4585178336899043  \\
            1.155140491216176  1.440234464769572  \\
            1.1587752307363617  1.42211767311529  \\
            1.1633610248895134  1.404217953474771  \\
            1.168885092244832  1.3865851955770918  \\
            1.1753320362358184  1.369268545081071  \\
            1.1826838880732655  1.352316266597502  \\
            1.1909201568274974  1.335775609167027  \\
            1.2000178865402633  1.3196926745685864  \\
            1.2099517202071135  1.3041122888254923  \\
            1.2206939704519162  1.2890778772672604  \\
            1.2322146966965422  1.2746313434954295  \\
            1.2444817886106254  1.260812952590708  \\
            1.2574610556088135  1.2476612188869713  \\
            1.271116322146061  1.2352127986249064  \\
            1.285409528545362  1.2235023877844944  \\
            1.3003008370768956  1.2125626253810904  \\
            1.3157487429929253  1.202424002494635  \\
            1.331710190208974  1.1931147772855455  \\
            1.3481406913088545  1.1846608962341534  \\
            1.364994451539077  1.177085921823211  \\
            1.3822244964470414  1.170410966865026  \\
            1.3997828028072605  1.1646546356562661  \\
            1.4176204324707051  1.1598329721244482  \\
            1.4356876687642002  1.1559594151106345  \\
            1.453934155059712  1.1530447609129713  \\
            1.4723090351273014  1.151097133195471  \\
            1.4907610948805572  1.150121960345903  \\
            1.5092389051194426  1.150121960345903  \\
            1.5276909648726986  1.151097133195471  \\
            1.5460658449402878  1.1530447609129713  \\
            1.5643123312357996  1.1559594151106345  \\
            1.5823795675292947  1.1598329721244482  \\
            1.6002171971927392  1.1646546356562661  \\
            1.6177755035529586  1.170410966865026  \\
            1.6350055484609227  1.177085921823211  \\
            1.6518593086911453  1.1846608962341532  \\
            1.6682898097910257  1.1931147772855455  \\
            1.6842512570070745  1.202424002494635  \\
            1.6996991629231042  1.2125626253810902  \\
            1.714590471454638  1.2235023877844942  \\
            1.7288836778539387  1.2352127986249064  \\
            1.7425389443911863  1.2476612188869711  \\
            1.7555182113893744  1.260812952590708  \\
            1.7677853033034578  1.2746313434954295  \\
            1.7793060295480838  1.2890778772672602  \\
            1.7900482797928865  1.3041122888254921  \\
            1.7999821134597367  1.3196926745685864  \\
            1.8090798431725026  1.335775609167027  \\
            1.8173161119267343  1.3523162665975015  \\
            1.8246679637641816  1.3692685450810709  \\
            1.831114907755168  1.3865851955770918  \\
            1.8366389751104866  1.404217953474771  \\
            1.841224769263638  1.4221176731152898  \\
            1.844859508783824  1.4402344647695717  \\
            1.8475330630000857  1.458517833689904  \\
            1.8492379802373018  1.4769168208478567  \\
            1.8499695085853438  1.4953801449662345  \\
            1.8497256091435006  1.513856345449194  \\
            1.8485069617032621  1.5322939258121555  \\
            1.846316962853617  1.550641497211739  \\
            1.8431617165141503  1.5688479216756896  \\
            1.839050016922326  1.5868624546335783  \\
            1.833993324122368  1.6046348863510196  \\
            1.8280057320240628  1.6221156818732083  \\
            1.8211039291205047  1.6392561190877224  \\
            1.8133071519742718  1.656008424521788  \\
            1.804637131601677  1.6723259064955196  \\
            1.7951180329045267  1.688163085260002  \\
            1.784776387318206  1.7034758197575108  \\
            1.7736410188638105  1.7182214306505568  \\
            1.7617429638104276  1.7323588192768529  \\
            1.74911538417149  1.7458485821986593  \\
            1.735793475276295  1.7586531210272307  \\
            1.7218143676743105  1.7707367472162687  \\
            1.7072170236456736  1.782065781532302  \\
            1.692042128606326  1.7926086479247516  \\
            1.6763319777104655  1.8023359615340455  \\
            1.660130357966363  1.8112206105924933  \\
            1.6434824261941188  1.8192378319896456  \\
            1.6264345831655036  1.8263652802915245  \\
            1.609034344276687  1.8325830900213553  \\
            1.5913302071143058  1.8378739310282135  \\
            1.5733715162839916  1.8422230567892641  \\
            1.5552083258781033  1.8456183455109656  \\
        }
        ;
    \addplot[color={rgb,1:red,0.0;green,0.1804;blue,0.3647}, name path={84134bdd-fda6-48ad-802d-f19e6ad15bbf}, draw opacity={1.0}, line width={2.0}, solid, quiver={u={\thisrow{u}}, v={\thisrow{v}}, every arrow/.append style={-{stealth[length = 0.3pt, width = 0.3pt]}}}]
        table[row sep={\\}]
        {
            x  y  u  v  \\
            1.6104166517562066  2.191236691021931  -0.036634131824212934  0.004863976807431669  \\
        }
        ;
    \addplot[color={rgb,1:red,0.0;green,0.1804;blue,0.3647}, name path={84134bdd-fda6-48ad-802d-f19e6ad15bbf}, draw opacity={1.0}, line width={2.0}, solid]
        table[row sep={\\}]
        {
            \\
            1.5  2.2  \\
            1.4630572570086764  2.1990244872250733  \\
            1.4262174800680065  2.196100667829363  \\
            1.3895833482437938  2.191236691021931  \\
            1.3532569674320174  2.1844461135785287  \\
            1.3173395857713883  2.175747862056427  \\
            1.281931311446626  2.1651661800427107  \\
            1.247130833668993  2.152730560583049  \\
            1.2130351476117627  2.1384756639792917  \\
            1.1797392840672738  2.122441221184987  \\
            1.1473360445790695  2.1046719230680915  \\
            1.1159157427873485  2.0852172958495037  \\
            1.0855659527086532  2.0641315630646044  \\
            1.0563712646513788  2.0414734944325374  \\
            1.02841304944741  2.0173062420544614  \\
            1.0017692316570206  1.9916971643973191  \\
            0.976514072379145  1.964717638553706  \\
            0.9527179622723793  1.9364428613011135  \\
            0.9304472253635879  1.906951639515022  \\
            0.9097639341909469  1.876326170520004  \\
            0.8907257367966465  1.8446518129910396  \\
            0.8733856960514564  1.8120168490435768  \\
            0.8577921417589907  1.7785122381754446  \\
            0.8439885359518742  1.7442313637464166  \\
            0.832013351755264  1.7092697727020394  \\
            0.8218999661553481  1.6737249092671567  \\
            0.8136765669716993  1.6376958433513797  \\
            0.8073660742927664  1.6012829944234783  \\
            0.8029860765934759  1.5645878516243117  \\
            0.8005487817129987  1.5277126908983887  \\
            0.8000609828293126  1.4907602899324688  \\
            0.8015240395253963  1.4538336416957132  \\
            0.8049338739998285  1.4170356673798084  \\
            0.8102809824323518  1.380468929539144  \\
            0.8175504614727235  1.34423534623058  \\
            0.8267220497790267  1.308435906949542  \\
            0.8377701844896638  1.2731703911541838  \\
            0.8506640724716366  1.2385370901621422  \\
            0.8653677761465312  1.2046325331950036  \\
            0.8818403136549946  1.1715512183340537  \\
            0.9000357730805267  1.1393853491371728  \\
            0.9199034404142269  1.1082245776509845  \\
            0.9413879409038323  1.0781557545345206  \\
            0.9644293933930842  1.049262686990859  \\
            0.988963577221251  1.021625905181416  \\
            1.0149221112176272  0.9953224377739428  \\
            1.0422326442921221  0.970425597249813  \\
            1.070819057090724  0.9470047755689888  \\
            1.1006016741537912  0.9251252507621808  \\
            1.1314974859858506  0.9048480049892702  \\
            1.1634203804179484  0.8862295545710911  \\
            1.1962813826177092  0.8693217924683068  \\
            1.2299889030781541  0.8541718436464221  \\
            1.2644489928940825  0.840821933730052  \\
            1.2995656056145213  0.8293092713125324  \\
            1.3352408649414105  0.8196659442488967  \\
            1.3713753375284004  0.8119188302212689  \\
            1.4078683101194243  0.8060895218259423  \\
            1.4446180702546025  0.8021942663909418  \\
            1.4815221897611144  0.8002439206918058  \\
            1.5184778102388854  0.8002439206918058  \\
            1.5553819297453972  0.8021942663909418  \\
            1.5921316898805755  0.8060895218259423  \\
            1.6286246624715994  0.8119188302212688  \\
            1.6647591350585893  0.8196659442488966  \\
            1.7004343943854785  0.8293092713125323  \\
            1.7355510071059173  0.840821933730052  \\
            1.7700110969218454  0.854171843646422  \\
            1.8037186173822906  0.8693217924683065  \\
            1.8365796195820514  0.8862295545710909  \\
            1.8685025140141491  0.9048480049892702  \\
            1.8993983258462086  0.9251252507621806  \\
            1.929180942909276  0.9470047755689885  \\
            1.9577673557078774  0.9704255972498127  \\
            1.9850778887823726  0.9953224377739425  \\
            2.0110364227787487  1.0216259051814158  \\
            2.0355706066069157  1.049262686990859  \\
            2.0586120590961676  1.0781557545345204  \\
            2.080096559585773  1.1082245776509843  \\
            2.0999642269194734  1.1393853491371728  \\
            2.118159686345005  1.1715512183340537  \\
            2.1346322238534685  1.2046325331950032  \\
            2.149335927528363  1.2385370901621415  \\
            2.162229815510336  1.2731703911541836  \\
            2.173277950220973  1.3084359069495417  \\
            2.182449538527276  1.3442353462305798  \\
            2.189719017567648  1.3804689295391437  \\
            2.1950661260001714  1.4170356673798081  \\
            2.1984759604746036  1.4538336416957134  \\
            2.1999390171706876  1.490760289932469  \\
            2.199451218287001  1.5277126908983882  \\
            2.1970139234065242  1.564587851624311  \\
            2.192633925707234  1.6012829944234779  \\
            2.1863234330283006  1.6376958433513795  \\
            2.178100033844652  1.6737249092671564  \\
            2.167986648244736  1.7092697727020392  \\
            2.1560114640481256  1.7442313637464166  \\
            2.1422078582410093  1.7785122381754446  \\
            2.1266143039485437  1.8120168490435764  \\
            2.109274263203354  1.8446518129910392  \\
            2.0902360658090533  1.8763261705200038  \\
            2.069552774636412  1.9069516395150217  \\
            2.047282037727621  1.9364428613011133  \\
            2.023485927620855  1.9647176385537057  \\
            1.9982307683429796  1.991697164397319  \\
            1.97158695055259  2.0173062420544614  \\
            1.9436287353486212  2.0414734944325374  \\
            1.9144340472913473  2.064131563064604  \\
            1.884084257212652  2.085217295849503  \\
            1.852663955420931  2.104671923068091  \\
            1.8202607159327264  2.1224412211849866  \\
            1.7869648523882375  2.1384756639792912  \\
            1.7528691663310072  2.152730560583049  \\
            1.718068688553374  2.1651661800427107  \\
            1.6826604142286117  2.175747862056427  \\
            1.646743032567983  2.1844461135785282  \\
            1.6104166517562066  2.191236691021931  \\
        }
        ;
    \addplot[color={rgb,1:red,0.0;green,0.1804;blue,0.3647}, name path={6a0e4139-3b69-4f8c-ad57-8feefb705e28}, draw opacity={1.0}, line width={2.0}, solid, quiver={u={\thisrow{u}}, v={\thisrow{v}}, every arrow/.append style={-{stealth[length = 0.3pt, width = 0.3pt]}}}]
        table[row sep={\\}]
        {
            x  y  u  v  \\
            1.66562497763431  2.5368550365328963  -0.05495119773631929  0.007295965211147948  \\
        }
        ;
    \addplot[color={rgb,1:red,0.0;green,0.1804;blue,0.3647}, name path={6a0e4139-3b69-4f8c-ad57-8feefb705e28}, draw opacity={1.0}, line width={2.0}, solid]
        table[row sep={\\}]
        {
            \\
            1.5  2.55  \\
            1.4445858855130145  2.5485367308376103  \\
            1.38932622010201  2.5441510017440443  \\
            1.3343750223656907  2.536855036532897  \\
            1.2798854511480262  2.526669170367793  \\
            1.2260093786570825  2.5136217930846403  \\
            1.172896967169939  2.4977492700640656  \\
            1.1206962505034894  2.4790958408745736  \\
            1.069552721417644  2.457713495968937  \\
            1.019608926100911  2.43366183177748  \\
            0.9710040668686041  2.407007884602137  \\
            0.9238736141810228  2.3778259437742557  \\
            0.87834892906298  2.3461973445969067  \\
            0.8345568969770681  2.3122102416488057  \\
            0.7926195741711151  2.275959363081692  \\
            0.7526538474855308  2.2375457465959787  \\
            0.7147711085687176  2.197076457830559  \\
            0.679076943408569  2.1546642919516703  \\
            0.645670838045382  2.1104274592725325  \\
            0.6146459012864205  2.064489255780006  \\
            0.5860886051949697  2.0169777194865595  \\
            0.5600785440771848  1.9680252735653652  \\
            0.5366882126384861  1.9177683572631667  \\
            0.5159828039278114  1.866347045619625  \\
            0.4980200276328961  1.8139046590530592  \\
            0.48284994923302227  1.7605873639007352  \\
            0.47051485045754915  1.7065437650270696  \\
            0.46104911143914973  1.6519244916352174  \\
            0.4544791148902141  1.5968817774364674  \\
            0.4508231725694982  1.5415690363475831  \\
            0.450091474243969  1.4861404348987033  \\
            0.4522860592880946  1.4307504625435699  \\
            0.45740081099974295  1.3755535010697126  \\
            0.4654214736485278  1.3207033943087159  \\
            0.47632569220908527  1.26635301934587  \\
            0.49008307466854006  1.2126538604243131  \\
            0.5066552767344957  1.1597555867312757  \\
            0.5259961087074548  1.107805635243213  \\
            0.5480516642197969  1.0569487997925056  \\
            0.572760470482492  1.0073268275010807  \\
            0.6000536596207903  0.9590780237057591  \\
            0.6298551606213404  0.9123368664764767  \\
            0.6620819113557486  0.8672336318017811  \\
            0.6966440900896265  0.8238940304862887  \\
            0.7334453658318766  0.7824388577721241  \\
            0.7723831668264408  0.7429836566609144  \\
            0.8133489664381834  0.7056383958747195  \\
            0.8562285856360858  0.6705071633534831  \\
            0.9009025112306869  0.6376878761432712  \\
            0.9472462289787762  0.6072720074839054  \\
            0.9951305706269226  0.5793443318566367  \\
            1.044422073926564  0.5539826887024601  \\
            1.0949833546172314  0.5312577654696331  \\
            1.146673489341124  0.5112329005950781  \\
            1.1993484084217818  0.49396390696879866  \\
            1.2528612974121156  0.4794989163733452  \\
            1.3070630062926005  0.4678782453319035  \\
            1.3618024651791365  0.45913428273891355  \\
            1.416927105381904  0.45329139958641296  \\
            1.4722832846416716  0.4503658810377089  \\
            1.527716715358328  0.4503658810377089  \\
            1.5830728946180956  0.45329139958641296  \\
            1.6381975348208633  0.45913428273891355  \\
            1.692936993707399  0.46787824533190325  \\
            1.747138702587884  0.47949891637334496  \\
            1.8006515915782177  0.49396390696879866  \\
            1.8533265106588757  0.511232900595078  \\
            1.9050166453827682  0.531257765469633  \\
            1.9555779260734358  0.5539826887024599  \\
            2.0048694293730773  0.5793443318566365  \\
            2.052753771021224  0.6072720074839053  \\
            2.0990974887693126  0.6376878761432709  \\
            2.143771414363914  0.6705071633534829  \\
            2.186651033561816  0.7056383958747191  \\
            2.227616833173559  0.7429836566609137  \\
            2.266554634168123  0.7824388577721237  \\
            2.3033559099103735  0.8238940304862885  \\
            2.337918088644251  0.8672336318017808  \\
            2.3701448393786593  0.9123368664764764  \\
            2.3999463403792096  0.9590780237057592  \\
            2.427239529517508  1.0073268275010807  \\
            2.4519483357802025  1.056948799792505  \\
            2.474003891292545  1.1078056352432124  \\
            2.493344723265504  1.1597555867312754  \\
            2.50991692533146  1.2126538604243127  \\
            2.5236743077909143  1.2663530193458696  \\
            2.5345785263514724  1.3207033943087154  \\
            2.542599189000257  1.3755535010697122  \\
            2.5477139407119056  1.4307504625435699  \\
            2.549908525756031  1.4861404348987033  \\
            2.549176827430502  1.5415690363475822  \\
            2.545520885109786  1.5968817774364665  \\
            2.5389508885608505  1.651924491635217  \\
            2.5294851495424506  1.7065437650270692  \\
            2.517150050766978  1.7605873639007348  \\
            2.501979972367104  1.8139046590530588  \\
            2.4840171960721884  1.866347045619625  \\
            2.463311787361514  1.917768357263167  \\
            2.4399214559228155  1.9680252735653645  \\
            2.4139113948050306  2.0169777194865586  \\
            2.38535409871358  2.0644892557800056  \\
            2.3543291619546185  2.1104274592725325  \\
            2.320923056591431  2.15466429195167  \\
            2.2852288914312826  2.1970764578305584  \\
            2.2473461525144693  2.2375457465959783  \\
            2.2073804258288847  2.275959363081692  \\
            2.1654431030229317  2.3122102416488057  \\
            2.1216510709370207  2.346197344596906  \\
            2.076126385818978  2.3778259437742553  \\
            2.0289959331313963  2.4070078846021365  \\
            1.9803910738990895  2.43366183177748  \\
            1.9304472785823563  2.457713495968937  \\
            1.8793037494965108  2.4790958408745736  \\
            1.827103032830061  2.4977492700640656  \\
            1.7739906213429175  2.5136217930846403  \\
            1.7201145488519747  2.5266691703677924  \\
            1.66562497763431  2.5368550365328963  \\
        }
        ;
    \addplot[color={rgb,1:red,0.0;green,0.1804;blue,0.3647}, name path={642a1ea0-7059-40fb-a7f3-27e145b8faf2}, draw opacity={1.0}, line width={2.0}, solid, quiver={u={\thisrow{u}}, v={\thisrow{v}}, every arrow/.append style={-{stealth[length = 0.3pt, width = 0.3pt]}}}]
        table[row sep={\\}]
        {
            x  y  u  v  \\
            1.7208333035124133  2.882473382043862  -0.07326826364842565  0.009727953614863782  \\
        }
        ;
    \addplot[color={rgb,1:red,0.0;green,0.1804;blue,0.3647}, name path={642a1ea0-7059-40fb-a7f3-27e145b8faf2}, draw opacity={1.0}, line width={2.0}, solid]
        table[row sep={\\}]
        {
            \\
            1.5  2.9  \\
            1.4261145140173526  2.898048974450147  \\
            1.3524349601360133  2.8922013356587257  \\
            1.2791666964875874  2.8824733820438624  \\
            1.206513934864035  2.8688922271570574  \\
            1.1346791715427766  2.851495724112854  \\
            1.063862622893252  2.830332360085421  \\
            0.9942616673379859  2.805461121166098  \\
            0.9260702952235252  2.776951327958583  \\
            0.8594785681345477  2.7448824423699736  \\
            0.7946720891581387  2.709343846136183  \\
            0.731831485574697  2.6704345916990073  \\
            0.6711319054173066  2.628263126129209  \\
            0.6127425293027574  2.5829469888650745  \\
            0.5568260988948199  2.534612484108923  \\
            0.503538463314041  2.4833943287946383  \\
            0.45302814475829  2.429435277107412  \\
            0.4054359245447585  2.372885722602227  \\
            0.36089445072717585  2.313903279030044  \\
            0.31952786838189384  2.252652341040008  \\
            0.2814514735932929  2.1893036259820793  \\
            0.24677139210291288  2.1240336980871537  \\
            0.2155842835179813  2.057024476350889  \\
            0.18797707190374835  1.9884627274928333  \\
            0.16402670351052806  1.9185395454040788  \\
            0.1437999323106962  1.8474498185343136  \\
            0.12735313394339864  1.7753916867027595  \\
            0.11473214858553282  1.7025659888469564  \\
            0.10597215318695175  1.6291757032486232  \\
            0.1010975634259974  1.5554253817967776  \\
            0.10012196565862519  1.4815205798649378  \\
            0.10304807905079261  1.4076672833914265  \\
            0.10986774799965704  1.3340713347596167  \\
            0.12056196486470361  1.2609378590782878  \\
            0.13510092294544696  1.18847069246116  \\
            0.1534440995580535  1.116871813899084  \\
            0.17554036897932757  1.0463407823083677  \\
            0.2013281449432731  0.9770741803242842  \\
            0.2307355522930623  0.9092650663900074  \\
            0.2636806273099892  0.8431024366681075  \\
            0.30007154616105347  0.7787706982743454  \\
            0.3398068808284538  0.716449155301969  \\
            0.38277588180766453  0.6563115090690413  \\
            0.4288587867861684  0.5985253739817182  \\
            0.4779271544425019  0.543251810362832  \\
            0.5298442224352543  0.4906448755478856  \\
            0.5844652885842445  0.44085119449962606  \\
            0.6416381141814477  0.39400955113797753  \\
            0.7012033483075825  0.35025050152436155  \\
            0.7629949719717015  0.3096960099785404  \\
            0.8268407608358966  0.2724591091421822  \\
            0.8925627652354184  0.23864358493661353  \\
            0.9599778061563085  0.20834368729284414  \\
            1.028897985788165  0.18164386746010397  \\
            1.0991312112290426  0.1586185426250648  \\
            1.170481729882821  0.13933188849779343  \\
            1.2427506750568007  0.12383766044253774  \\
            1.3157366202388485  0.11217904365188458  \\
            1.389236140509205  0.10438853278188365  \\
            1.4630443795222288  0.10048784138361166  \\
            1.5369556204777708  0.10048784138361166  \\
            1.6107638594907943  0.10438853278188365  \\
            1.684263379761151  0.11217904365188458  \\
            1.7572493249431986  0.12383766044253752  \\
            1.8295182701171786  0.1393318884977932  \\
            1.900868788770957  0.15861854262506458  \\
            1.9711020142118343  0.18164386746010397  \\
            2.040022193843691  0.20834368729284392  \\
            2.107437234764581  0.23864358493661308  \\
            2.173159239164103  0.27245910914218174  \\
            2.2370050280282983  0.3096960099785404  \\
            2.298796651692417  0.3502505015243611  \\
            2.358361885818552  0.3940095511379771  \\
            2.415534711415755  0.4408511944996254  \\
            2.470155777564745  0.49064487554788494  \\
            2.5220728455574974  0.5432518103628317  \\
            2.5711412132138314  0.5985253739817179  \\
            2.6172241181923352  0.6563115090690409  \\
            2.660193119171546  0.7164491553019684  \\
            2.6999284538389468  0.7787706982743456  \\
            2.736319372690011  0.8431024366681076  \\
            2.769264447706937  0.9092650663900064  \\
            2.7986718550567264  0.9770741803242831  \\
            2.824459631020672  1.0463407823083672  \\
            2.8465559004419463  1.1168718138990836  \\
            2.864899077054553  1.1884706924611594  \\
            2.8794380351352964  1.2609378590782874  \\
            2.8901322520003427  1.3340713347596163  \\
            2.896951920949207  1.4076672833914265  \\
            2.899878034341375  1.4815205798649378  \\
            2.8989024365740024  1.5554253817967765  \\
            2.8940278468130485  1.629175703248622  \\
            2.8852678514144676  1.702565988846956  \\
            2.872646866056601  1.775391686702759  \\
            2.8562000676893042  1.8474498185343131  \\
            2.835973296489472  1.9185395454040783  \\
            2.8120229280962517  1.9884627274928333  \\
            2.7844157164820187  2.057024476350889  \\
            2.7532286078970873  2.124033698087153  \\
            2.7185485264067077  2.1893036259820784  \\
            2.6804721316181066  2.2526523410400077  \\
            2.6391055492728244  2.3139032790300433  \\
            2.5945640754552417  2.3728857226022266  \\
            2.5469718552417104  2.4294352771074115  \\
            2.4964615366859593  2.483394328794638  \\
            2.44317390110518  2.534612484108923  \\
            2.3872574706972425  2.5829469888650745  \\
            2.3288680945826945  2.628263126129208  \\
            2.268168514425304  2.670434591699007  \\
            2.205327910841862  2.709343846136182  \\
            2.140521431865453  2.744882442369973  \\
            2.073929704776475  2.7769513279585825  \\
            2.0057383326620144  2.805461121166098  \\
            1.936137377106748  2.830332360085421  \\
            1.8653208284572234  2.851495724112854  \\
            1.793486065135966  2.8688922271570565  \\
            1.7208333035124133  2.882473382043862  \\
        }
        ;
\end{axis}
\end{tikzpicture}

\end{marginfigure}

\begin{equation}
	\label{eqn:vortexflow}
	\phi_v = \frac{\Gamma}{2\pi} \theta
\end{equation}

\noindent where \(\Gamma\) is the vortex strength.


\subsection{Superposition Examples}

\toadd{add in rankine oval example, and combine with non-lifting cylinder}
\subsubsection{Rankine Oval}
As a preview to next chapter, an interesting superposition of a uniform flow, a source, and a sink results in an oval shaped ``bubble'' in the flow field.

[todo: explain the maths and how it's just addition.]

[Also probably want to talk about how to get velocity or stream or something for plotting/postprocessing and show some figures.]

\toadd{add in lifting cylinder example to show how vortices are superimposed}
\subsubsection{Lifting Cylinder}
