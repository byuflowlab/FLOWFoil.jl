\section{Introduction}
\label{sec:potentialflowintro}

This, and much of the following, chapter is built upon the simple equation:

\[\nabla^2\phi=0.\]

\noindent This equation is commonly known as Laplace's equation, and, among other things, is the governing equation of fluid flows that are both incompressible and irrotational:

\[
\begin{aligned}
	\nabla\cdot\vect{V} = 0,&~~~~~\mathrm{Incompressible}\\
	\nabla\times\vect{V} = 0,&~~~~~\mathrm{Irrotational}
\end{aligned}
\]

\noindent In the context presented here, the scalar field, \(\phi\), in Laplace's equation is called the velocity potential, or simply potential.
Mathematically, this means

\[\vect{V} = \nabla \phi.\]

\noindent By solving Laplace's equation for the velocity potential, we can thereby obtain the velocity of the flow, which in turn allows us to calculate a variety of useful characteristics about the flow.
It is for this reason that we often call inviscid, irrotational fluid flow, potential flow.
The rest of this chapter will cover the basics of potential flow theory that are applicable to the contents of this dissertation.



