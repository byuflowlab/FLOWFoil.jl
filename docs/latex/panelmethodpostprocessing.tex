\section{Panel Method Post Processing}
\label{sec:panelmethodpostprocessing}

\subsection{Surface Velocity}

For the inviscid panel methods using vortex distributions, the vortex strengths solved for in the linear system are synonymous with the surface velocity magnitudes.

\subsection{Surface Pressure}

The pressure coefficient is defined as

\begin{equation}
	C_p = \frac{p_1 - p_\infty}{\frac{1}{2} \rho_\infty V_\infty^2}
\end{equation}

\noindent where \(p = p_1-p_\infty\) is the gage pressure, and \(\rho_\infty\) and \(V_\infty\) are the freestream density and velocity, respectively.

If we assume incompressible and irrotational flow conditions\sidenote{along with the other four required assumptions for Bernoulli's equation}, the we can apply Bernoulli's equation to simplify our expression for the pressure coefficient by putting things in terms of the gage pressure:

\begin{equation}
\begin{aligned}
	p_\infty + \frac{1}{2} \rho_\infty V_\infty^2 &= p_1 + \frac{1}{2} \rho_\infty V^2 \\
	p_1 - p_\infty  &= \frac{1}{2} \rho_\infty V_\infty^2 - \frac{1}{2} \rho_\infty V^2,
\end{aligned}
\end{equation}

\noindent then substituting into our coefficient expression and simplifying:

\begin{equation}
\begin{aligned}
	C_p =& \frac{\frac{1}{2} \rho_\infty V_\infty^2 - \frac{1}{2} \rho_\infty V^2}{\frac{1}{2} \rho_\infty V_\infty^2} \\
	C_p =& \frac{\frac{1}{2} \rho_\infty V_\infty^2}{\frac{1}{2} \rho_\infty V_\infty^2} - \frac{\frac{1}{2} \rho_\infty V^2}{\frac{1}{2} \rho_\infty V_\infty^2} \\
	C_p =& 1 - \left(\frac{V}{V_\infty}\right)^2.
\end{aligned}
\end{equation}

\noindent Thus we can use the surface velocity as \(V\) and find the surface pressure of the body in question (usually an airfoil).

\subsection{Forces and Moments}

We can integrate the inviscid surface pressure to find forces, such as lift, inviscid drag, and moments.
After we have calculated the surface pressure coefficients for each panel using the panel surface velocity, we can dimensionalize the pressure in preparation to get the dimensional loads.

\begin{equation}
	p = C_p \frac{1}{2} \rho_\infty V_\infty^2
\end{equation}

We also know that pressure acts normal to the surface, and we know the normal vectors associated with each panel.
By integrating the pressure over the surface, taking into account the panel directional components, we can obtain the loads on the body.

\begin{equation}
	\begin{aligned}
	F'_x &= \int_s -p \hat{n}_x \d s   \\
	F'_y &= \int_s p \hat{n}_y \d s 
	\end{aligned}
\end{equation}

\noindent Since we have used flat panels for the methods presented thus far, we can evaluate the integral using a sum:

\begin{equation}
	\begin{aligned}
		f'_x &= \sum_i^N -p_{i} \hat{n}_{x_i} d_i  \\
	 	f'_y &= \sum_i^N p_{i} \hat{n}_{y_i} d_i
	\end{aligned}
\end{equation}

For the axisymmetric, annular airfoil case, we may also want to find the total duct loads.
For this case, we use cylindrical coordinates, so the \(y\) subscripts become \(r\)'s.
We can immediately see that by symmetry the integral around the axis of rotation yields a net zero load in the radial direction.
For the axial direction (still the x-direction) we have

\begin{equation}
	\begin{aligned}
		F_x &= \int_s \int_0^{2\pi} -p \hat{n}_x r \d \theta \d s \\
		F_x &= 2\pi \sum_i^N -p_{i} \hat{n}_{x_i} r_i d_i ,
	\end{aligned}
\end{equation}

\noindent where \(r_i\) is the radial position of the ith panel control point, relative to the axis of rotation.
Since the x-direction is traditionally downstream, if we want to find the thrust, \(T\), of an annular airfoil (duct) we simply take the negative of the axial force, thus \(T = -F_x\).
