\documentclass[]{article}

%Move Title Up
\usepackage[T1]{fontenc}
\usepackage{titling}
\setlength{\droptitle}{-10em}   % This is your set screw

%Date Format
\usepackage[]{datetime}
\newdateformat{mydate}{\THEDAY~\monthname[\THEMONTH]~\THEYEAR}

%Bullet Style
%\renewcommand{\labelitemi}{-}


%graphics packages
\usepackage[]{graphicx}
\graphicspath{{./figures/}}
\usepackage{caption}
\usepackage{subcaption}
\usepackage{placeins}
\usepackage{wrapfig}


%mathpackages
\usepackage{amsmath}
\usepackage{amsfonts} 
\usepackage{siunitx}
\usepackage{mathrsfs}
\newcommand{\vect}{\mathbf}
%\newcommand{\vect}{\overset{\rightharpoonup}}
\usepackage{nicefrac}
\usepackage{cancel}

%referencing packages
\usepackage{hyperref}
\usepackage[noabbrev]{cleveref}



\title{FLOWFoil Theory Document}

% Your name goes here:
\author{Judd Mehr, }
\date{}



\begin{document}

\maketitle

\noindent \textbf{Last Updated:} \mydate\today

%%%%%%%%%%%%%%%%%%%%%%%%%%%%
%%%%%%%%%%%%%%%%%%%%%%%%%%%%
\section{Geometry}
\label{sec:geometry}

\subsection{Panels}
\label{ssec:panels}

For any geometric input, we assume that the airfoil geometry begins at the trailing egde and proceeds clockwise around the leading edge and back to the trailing edge.
It is likely convenient to input geometry data that positions the airfoil leading edge at the origin and has a chord length of one.

Linear panels are defines using the input coordinates as the panel end points.
\Cref{fig:linearpanels} shows the geometry convention uses in FLOWFoil as well as the linear paneling of an airfoil.

\begin{figure}[h]
	\centering
	\includegraphics[width=0.5\textwidth]{draft}
	\caption{Linear Paneling}
	\label{fig:linearpanels}
\end{figure}


\section{Inviscid Solution Details}
\label{sec:inviscidsolution}


\subsection{Vortex Influence Coefficients}
\label{ssec:vortexinfluencecoefficients}

Following the formulation in \cite{fidkowski_coupled_2022}, the inviscid system is assembled as follows.

\begin{equation}
	\mathbf{A} \mathbf{\gamma} = \mathbf{\Psi}^\infty
\end{equation}

\noindent where \(\mathbf{A}\) is comprized of the coefficients of influence between the panels and evaluation points.

For a given evaluation point, \(i \in \mathbb{N}^N\), and panel, \(k \in \mathbb{N}^{N-1}\), (comprised of nodes \(j \in \mathbb{N}^N\) and \(j+1\)) the streamfunction at the evaluation point due to a linear vortex distribution across the panel (with vortex strengths at the nodes of \(\gamma_j\) and \(\gamma_{j+1}\), respectively) is

\begin{equation}
	\Psi_{ik} = \mathbf{\Psi}^\gamma_{ik} [\gamma_j, \gamma_{j+1}]^\top,
\end{equation}

\noindent where

\begin{equation}
	\mathbf{\Psi}^\gamma_{ik} = [\overline{\psi}^\gamma_{ik} - \widetilde{\psi}^\gamma_{ik}, \widetilde{\psi}^\gamma_{ik}]
\end{equation}

\noindent Therefore, the influence coefficient of the \(j\)th node on the \(i\)th node (the \(ij\)th element of \(\mathbf{A}\)) is comprized of portions of the influence seen from each panel of which it is part:

\begin{equation}
	A_{ij} = \begin{cases}
		\overline{\psi}^\gamma_{ij} -  \widetilde{\psi}^\gamma_{ij} & j = 1 \\
		\widetilde{\psi}^\gamma_{i,j-1} & j = N \\
		\widetilde{\psi}^\gamma_{i,j-1} + \overline{\psi}^\gamma_{ij} -  \widetilde{\psi}^\gamma_{ij} & \mathrm{otherwise}.
	\end{cases}
\end{equation}

\subsubsection{Influence Geometry}

The components of \(\mathbf{\Psi}^\gamma_{ik}\) are defined as 

\begin{align}
	\overline{\psi}^\gamma_{ik} &= \frac{1}{2\pi} \left( h (\theta_{j+1} - \theta{j}) - d + a \ln(r_j) - (a-d)\ln(r_j+1) \right) \\
	\widetilde{\psi}^\gamma_{ik} &= \frac{a}{d}\overline{\psi}^\gamma_{ij} + \frac{1}{4\pi d} \left(r^2_{j+1} \ln(r_{j+1}) - r^2_{j} \ln(r_j) - \frac{1}{2}r^2_{j+1} + \frac{1}{2}r^2_j \right).
\end{align}
\noindent In order to evaluate the the componenets of \(\mathbf{\Psi}^\gamma_{ik}\) we will need to understand the geometry of the problem.

\begin{figure}[h]
	\centering
	\includegraphics[width=0.5\textwidth]{draft}
	\caption{Influence Geometry}
	\label{fig:influencegeometry}
\end{figure}

\Cref{fig:influencegeometry} shows the relative geometry for the problem, and we calculate each of the geometric values as follows.

\begin{itemize}
	\item The \(k\)th panel length, from the \(j\)th to \(j+1\)th node:
	\begin{equation}
		\vect{d}_k = \vect{n}_{j+1} - \vect{n}_j
	\end{equation}
	\noindent where \(n\) is the node position.
	%
	%
	\item The vector from the \(j\)th node to the evaluation point:
	\begin{equation}
		\vect{r}_j = \vect{p} - \vect{n}_j
	\end{equation}
	\noindent where \(p\) is the evaluation point position.
	%
	%
	\item The angle between the \(k\)th panel and the evaluation point centered at the \(j\)th node:
	\begin{equation}
		\theta_j = \cos^{-1}\left( \frac{\vect{r}_j \cdot \vect{d}_k}{r_jd_k} \right)
	\end{equation}
	\noindent where \(r = ||\vect{r}||\) and \(d = ||\vect{d}||\).
	%
	%
	\item The height of the triangle made from the evaluation point and \(j\)th and \(j+1\)th nodes:
	\begin{equation}
		h_k = 2A/d_k
	\end{equation}
	\noindent where \(A = \left[ s (s - d_k) (s - r_j) (s - r_{j+1}) \right]^{1/2} \), where \(s = (d_k + r_j + r_{j+1})/2\) (this is Heron's formula for the area of a triangle).
	%
	%
	\item The length of the right triangle with base aligned with the panel, and height, \(h_k\):
	\begin{equation}
		a_k = \begin{cases}
			r_j \cos(\pi - \theta_j) + d_k & \mathrm{if}~~ \theta_j > \frac{\pi}{2} \\
			d_k & \mathrm{if}~~ \theta_j = \frac{\pi}{2} \\
			r_j\cos(\theta_j) & \mathrm{otherwise} 
		\end{cases}
	\end{equation}
	%
	%
	\item The natural log of the distance between node and evaluation point:
	\begin{equation}
		\ln(r) = \begin{cases}
			0 & \mathrm{if}~~ r = 0 \\
			\ln(r) & \mathrm{otherwise}
		\end{cases}
	\end{equation}
\end{itemize}




















\newpage

\bibliography{ref}{}
\bibliographystyle{aiaa}
%%%%%%%%%%%%%%%%%%%%%%%%%%%%
%%%%%%%%%%%%%%%%%%%%%%%%%%%%

\vspace{1cm}
\hrule
\vspace{1cm}

\noindent  \textbf{Change Log:}

\begin{itemize}
	\item (date): change(s)
\end{itemize}



\end{document}
